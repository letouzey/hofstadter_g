\documentclass[a4paper,11pt]{article}

\usepackage[T1]{fontenc}
\usepackage{aeguill}
\usepackage[utf8x]{inputenc}
\usepackage[english]{babel}
\usepackage{hyperref}
\usepackage{amsfonts, amsthm, amsmath}
\usepackage{tikz-qtree}

\title{Hofstadter's Curious Function H,
       Link with a Morphic Word and a Jacobi-Perron Fractal}
\author{Pierre letouzey}
\date{\small
{\tt letouzey@irif.fr}\\
IRIF, Université Paris Cité \& INRIA Paris\\
Technical report, Sept. 2022, \href{http://creativecommons.org/licenses/by/4.0/}{license CC-BY}
}

\begin{document}
\newtheorem{theorem}{Theorem}
\newtheorem{definition}{Definition}
\maketitle

\newcommand{\TODO}{{\color{red} TODO}\ }

%%TODO refresh this doc
\newcommand{\docgen}[2]{\href{http://www.irif.fr/~letouzey/hofstadter_g/doc/#1.html#2}{\tt #1.v}}
\newcommand{\doc}[1]{\docgen{#1}{}}
\newcommand{\doclab}[2]{\docgen{#1}{\##2}}
\newcommand{\FG}{\ensuremath{\overline{G}}}
\newcommand{\fibrest}{\ensuremath{\Sigma A_i}}
\newcommand{\flip}{\textit{flip}}
\newcommand{\depth}{\textit{depth}}

\section{Introduction}
In a previous work \cite{??}, we studied during a Coq development the
$G$ function (sequence A5206 in OEIS\cite{OEIS-G})
introduced by Hofstadter in his famous ``Gödel, Escher, Bach'' book\cite{GEB}.
In particular we proved an answer to Hofstadter's ``problem for the curious reader'' about $G$.
We focus now on the $H$ function proposed by
Hofstadter just after $G$ in the same book. This $H$
function (sequence A5374 in OEIS\cite{OEIS-H})
has a third nested recursive call in its definition instead of just
two for $G$:
\begin{align*}
  H(0) &= 0 \\
  H(n) &= n - H(H(H(n-1))) ~~ \text{for}~n>0
\end{align*}

In this report, we show that $H$ has many properties in common with $G$,
especially some strong links with a Fibonacci-like sequence and with
an infinite regular tree. In the same way $G$ is related with the
Fibonacci infinite word, $H$ could also be related with a infinite
morphic word. More surprisingly, $H$ has an intimate link with a nice fractal
named Jacobi-Perron \cite{??}, which is related with the Rauzy fractal
\cite{??}. In fact, the analysis of the Tribonacci word by Rauzy
\cite{??} was reused and adapted here to a large extend. This allowed
us to prove the following fact, which is currently mentionned as a
conjecture on OEIS:
$$\forall n\in\mathbb{N},~H(n) = \lfloor \tau n \rfloor + 0~\text{or}~1 $$
where $\tau$ is the real root of $X^3+X-1=0$ (approximately $0.6823278$).

Just like our earlier work on $G$, we formalized and checked our
proofs about $H$ using the Coq proof assistant\cite{Coq}.
The detailed and machine-checked proofs can be found in the files
of this development\footnote{See \url{http://www.irif.fr/~letouzey/hofstadter_g}}
and can be re-checked by running Coq \cite{Coq} %%version 8.4
on it.
No prior knowledge of Coq is assumed here, on the contrary this
document has rather been a ``Coq-to-English'' translation
exercise for the author. Nonetheless, some proofs given in this
document are still quite sketchy: in this case, the interested
reader is encouraged to consult the Coq files given as references.


\section{Prior art and contributions}

\TODO

%% Most of the results proved here were already mentioned elsewhere,
%% e.g. on the OEIS pages for $G$ and $\FG$. These results were certified
%% in Coq without consulting the existing proofs in the literature, so
%% the proofs presented here might still be improved. To the best of
%% my knowledge, the main novelties of this development are:
%% \begin{enumerate}
%% \item
%%   A proof of the recursive
%% equation for $\FG$ which is currently mentioned as a conjecture in OEIS:

%% $$\forall n>3, \FG(n) = n+1 - \FG(1+\FG(n-1))$$

%% \item
%%   The statement and proof of another equation for $\FG$:

%% $$\forall n>3, \FG(\FG(n)) + \FG(n-1) = n$$

%% Although simpler than the previous one, this equation isn't
%% enough to characterize $\FG$ unless an extra monotonocity
%% condition on $\FG$ is assumed.

%% \item
%%   The statement and proof of a result comparing $\FG$ and $G$:
%% for all $n$, $\FG(n)$ is either $G(n)+1$ or $G(n)$, depending
%% of the shape of the Zeckendorf decomposition of $n$ as a sum
%% of Fibonacci numbers. More precisely, when this decomposition
%% of $n$ starts with $F_3$ and has a second smallest term $F_k$
%% with $k$ odd, then $\FG(n)=G(n)+1$, otherwise $\FG(n)=G(n)$.
%% Moreover these
%% specific numbers where $\FG$ and $G$ differ are separated by either
%% 5 or 8.

%% \item The studies of $G$ and $\FG$ ``derivatives''
%%   ($\Delta G(n) = G(n+1)-G(n)$ and similarly for $\Delta\FG$),
%%   leading to yet another characterization of $G$ and $\FG$.

%%\end{enumerate}

\section{Coq jargon}
For readers without prior knowledge
of Coq but curious enough to have a look at the actual files
of this development, here comes a few words about the Coq syntax.
\begin{itemize}
\item By default, we're manipulating natural numbers only, corresponding
 to Coq type {\tt nat}. The files \doc{Phi} and \doc{Lim} manipulate
 real numbers (Coq type {\tt R}) and even complex numbers (type {\tt
   C}). Note that in Coq 8.11 and earlier, these real numbers are
 not axiomatized anymore, but implemented internally as Dedekind
 cuts, relying only on four logical axioms (excluded middle,
 stronger excluded middle for negations, limited principle of
 omniscience, functional extensionality).

\item The symbol {\tt S} denotes the successor of natural numbers.
  Hence {\tt S($2$)} is the same as $3$.
\item As in many modern functional languages such as OCaml or Haskell,
  the usage is to skip parenthesis whenever possible, and
  in particular around atomic arguments of functions. Hence
  {\tt S(x)} will rather be written {\tt S x}.
\item The symbol {\tt pred} is the predecessor for natural numbers.
  In Coq, all functions must be total, and {\tt pred $0$} has been
  chosen to be equal to $0$. Similarly, the subtraction on
  natural numbers is rounded: {\tt $3$ - $5$ = $0$}.
\item \TODO Coq allows to define custom predicates to express various
  properties of numbers, lists, etc. These custom predicates are
  introduced by the keyword {\tt Inductive}, followed by some
  ``introduction'' rules for the new predicate. In this development
  we use capitalized names for these predicates (e.g. {\tt Delta},
  {\tt Low}, ...).
\item \TODO Coq also accepts new definition of recursive functions via
  the command {\tt Fixpoint}. But these definitions should satisfy
  some criterion to guarantee that the new function is well-defined
  and total. So in this development, we also need sometimes to
  define new functions via explicit justification of termination,
  see for instance {\tt norm} in \doclab{Fib}{norm}\ or {\tt g\_spec} in
  \doclab{FunG}{g\_spec}.
\end{itemize}

\section{Fibonacci-like numbers and decompositions}

This section corresponds to file \doc{GenFib}.

Just like function $G$ is related with Fibonacci numbers,
the $H$ function will have intimate connections with a Fibonacci-like sequence
called Narayana's cows sequence (sequence A930 in OEIS\cite{OEIS-Cow}):

$$A_0 = 1 $$
$$A_1 = 2 $$
$$A_2 = 3 $$
$$\forall n\ge 2,~~ A_{n+1} = A_{n}+A_{n-2}$$

For the usual Fibonacci sequence, the recursive equation adds
consecutive terms to get the next term in the sequence. Here, we now add
terms that are two-step apart. Note that our choice of initial values
differs from OEIS : instead of three initial $1$ terms, we used
$1$ $2$ $3$ to avoid redundancy,
leading to a $(A_n)$ sequence that is the one of OEIS without its
first two terms. The first twenty terms in this $(A_n)$ sequence are:
%
$$ 1 ~ 2 ~ 3 ~ 4 ~ 6 ~ 9 ~ 13 ~ 19 ~ 28 ~ 41 ~ 60 ~ 88 ~ 129 ~ 189 ~
277 ~ 406 ~ 595 ~ 872 ~ 1278 ~ 1873 $$

In Coq, we actually defined in {\tt GenFib} a more general sequence
{\tt A} with an extra parameter $k$, for which the recursive equation
adds earlier terms that are $k$-step apart.
For $k=1$ we hence retrieve the Fibonacci numbers (shifted), while the previous
Narayana sequence $A_n$ is {\tt (A 2 $n$)} in Coq.
We then proved in Coq a few basic properties :
strict positivity, strict monotony, etc.
We also prove the theorem {\tt A\_inv} which states
that any positive number can be bounded by
consecutive $A_n$ numbers :

\begin{theorem}[Fibonacci-like inverse {\tt A\_inv}]\label{fibinv}
$\forall n>0, \exists p, A_p \le n < A_{p+1}$.
\end{theorem}

\subsection{Fibonacci-like decompositions} The rest of the file \doc{GenFib}\
deals with the decompositions of natural numbers as sums of $A_i$
numbers. First note that we consider here these sums up to permutation of
their terms, while the Coq implementation actually chooses a specific
ordering of terms in the representation of the sum, either strictly
increasing or decreasing depending of the need.

\begin{definition}
 A decomposition $n = \fibrest$ is said to be \emph{canonical} if
 any two terms $A_i$ and $A_j$ in the decomposition have indexes $i$
 and $j$ that are apart by at least 3, i.e. $|i-j|\ge 3$.
 And a decomposition is said to be \emph{lax} if all indexes in the
 decomposition are apart by at least 2.
\end{definition}

For instance, $15$ admits the canonical decomposition
$A_1+A_6 = 2 + 13$, but also the lax decomposition $A_1+A_3+A_5
= 2 + 4 + 9$.
Said otherwise, in a canonical decomposition, no couple of terms could
be added via the recursive equation of $(A_n)$ to form a higher term.
This is still possible in a lax decomposition, which simply ensure
the lack of redundant or successive terms. Lax decompositions will
be crucial later when studying the effect of function $H$ on
decompositions, while remaining close enough to canonical decompositions.

On a technical level, we represented in Coq the decompositions
as sorted lists of ranks, and we used a predicate {\tt (Delta $p$ $l$)}
to express that any element in the list $l$ exceeds the previous
element by at least $p$. We hence use $p=3$ for canonical
decompositions, and $p=2$ for lax ones. See file
\doc{DeltaList}\ for the definition and properties of this
predicate {\tt Delta}.

Then we adapt Zeckendorf's theorem (actually discovered earlier by
Lekkerkerker) for this settings:

\begin{theorem}[Zeckendorf]\label{zeck}
Any natural number has a unique canonical decomposition as sum of
$(A_i)$ numbers.
\end{theorem}

\begin{proof}
The proof of this theorem is quite standard. First we build by strong
induction a canonical decomposition of $n$. For $n=0$ we take
the empty decomposition, whereas for
$n>0$ we take first the highest $A_p$ less or equal to $n$
(cf {\tt A\_inv} above), add it to the decomposition,
and then build recursively a decomposition of $n-A_p$.
Note that the choice criterion of $A_p$ implies $n < A_{p+1}$.
If $p\le 2$, then $A_{p+1} = 1 + A_p$ and hence $n$ is exactly $A_p$
and the decomposition stops there. Otherwise $p>2$ and
hence $n < A_p + A_{p-2}$ hence $n-A_p < A_{p-2}$. So the next term in
the decomposition will be at most $A_{p-3}$. Repeating this will hence
build a decomposition that is indeed canonical.

For proving the uniqueness, the key ingredient is that a canonical
sum cannot exceed the next $A_p$ number above it. For instance
$A_1+A_4+A_7 = 2+6+19 = 27 = A_8 - 1$.
\end{proof}

\begin{definition}
For a non-empty decomposition, its \emph{rank} is the rank $p$ of
the lowest $A_p$ number in this decomposition. By extension, for a
number $n\neq 0$, we define $rank(n)$ to be the rank of its (unique)
canonical decomposition.
\end{definition}
For instance $rank(15)=1$.

Any lax decomposition can be normalized into a canonical decomposition
(see Coq function {\tt norm}):

\begin{theorem}[normalization]\label{norm}
Consider a lax decomposition of $n$, made of $k$ terms and
whose rank is $r$. Then the unique canonical decomposition
of $n$ is made of no more than $k$ terms, and its rank can
be written $r+3p$ for some $p\ge 0$.
\end{theorem}
\begin{proof}
We simply add together the highest couple $A_i + A_{i-2}$ present in
the lax decomposition (if any), replacing these two terms by
solely $A_{i+1}$. Then we repeat this process.
By dealing with highest terms first, we ensure that the decomposition
stays a lax one all along : in particular, $A_{i+2}$ was not in
the initial decomposition otherwise the couple $A_{i+2} + A_{i}$ would have
been considered first. The number of
terms is decreasing by 1 at each step, so the process is guaranteed to
terminate, and will necessarily stops on the canonical decomposition
of $n$ (no remaining terms to add).
And finally, it is easy to see that the
lowest rank is either left intact or raised by 3 at each step of the process.
\end{proof}

\subsection{Classifications of Fibonacci decompositions.}

\TODO Is it interesting ? Maybe for flip-H ? \\
\TODO Mention decomp of successor and predecessor ?


%% The end of \doc{Fib}\ deals with classifications of numbers
%% according to the lowest rank of their canonical decomposition.
%% In particular, this lowest rank could be 2, 3 or more. It will
%% also be interesting to distinguish between lowest ranks that
%% are even or odd. These kind of classifications and their
%% properties will be heavily used
%% during theorem \ref{comp-fg-g} which compares $\FG$ and $G$.
%% For instance:
%% \begin{theorem}[decomposition of successor]\label{fibsucc}
%% \noindent
%% \begin{enumerate}
%% \item $low(n) = 2$ implies $low(n+1)$ is odd.
%% \item $low(n) = 3$ implies $low(n+1)$ is even and different from 2.
%% \item $low(n) > 3$ implies $low(n+1) = 2$.
%% \end{enumerate}
%% \end{theorem}
%% \begin{proof}
%% Let $r$ be $low(n)$. We can write $n = F_r + \fibrest$ in a canonical
%% way.
%% \begin{enumerate}
%% \item If $r = 2$, then $n+1 = F_3 + \fibrest$ is a lax decomposition,
%%   and we conclude thanks to the previous normalization theorem.
%% \item If $r = 3$, then $n+1 = F_4 + \fibrest$ is a lax decomposition,
%%   and we conclude similarly.
%% \item If $r > 3$, then $n+1 = F_2 + F_r + \fibrest$ is directly a
%%   canonical decomposition.
%% \end{enumerate}
%% \end{proof}

%% We also studied the decomposition of the predecessor
%% $n-1$ in function of $n$. This decomposition depends of the parity of
%% the lowest rank in $n$, since for $r\neq 0$ we have:

%% $$ F_{2r} - 1 = F_3 + ... + F_{2r-1}$$
%% $$ F_{2r+1} - 1 = F_2 + F_4 + ... + F_{2r}$$

%% Hence:

%% \begin{theorem}[decomposition of predecessor]\label{fibpred}
%% \noindent
%% \begin{enumerate}
%% \item If $low(n)$ is odd then $low(n-1) = 2$.
%% \item If $low(n)$ is even and different from 2, then $low(n-1) = 3$.
%% \item For $n>1$, if $low(n)=2$ then $low(n-1)>3$.
%% \end{enumerate}
%% \end{theorem}
%% \begin{proof}
%% \noindent
%% \begin{enumerate}
%% \item Let the canonical decomposition of $n$ be $F_{2r+1} +
%%   \fibrest$. Then we have $n-1 = F_2 + ... + F_{2r} + \fibrest$, which
%%  is also a canonical decomposition. Moreover $r\neq 0$ since $F_1$
%%  isn't allowed in decompositions. Hence the decomposition of $n-1$
%%  contains at least the term $F_2$, and $low(n-1)=2$.
%% \item When $low(n) = 2r$ with $r\neq 0$, we decompose
%%  similarly $F_{2r}-1$, leading to a lowest rank 3 for $n-1$.
%% \item Finally, when $n = F_2 + \fibrest$, then the canonical decomposition
%%   of $n-1$ is directly the rest of the decomposition of $n$, which
%%   isn't empty (thanks to the condition $n>1$) and
%%   cannot starts by $F_2$ nor $F_3$ (for canonicity reasons).
%% \end{enumerate}
%% \end{proof}

%% \subsection{Subdivision of decompositions starting by $F_3$}

%% When $low(n)=3$ and $n>2$, the canonical decomposition of $n$
%% admits at least a second-lowest term: $n = F_3 + F_k + ...$ and we
%% will sometimes
%% need to consider the parity of this second-lowest rank.

%% \begin{definition}
%% \noindent
%% \begin{itemize}
%% \item A number $n$ is said \emph{3-even} when its canonical decomposition
%%   is of the form $F_3 + F_{2p} + ...$.
%% \item A number $n$ is said \emph{3-odd} when its canonical decomposition
%%   is of the form $F_3 + F_{2p+1} + ...$
%% \end{itemize}
%% \end{definition}
%% In Coq, this corresponds to the {\tt ThreeEven} and {\tt ThreeOdd}
%% predicates.
%% In fact, the 3-odd numbers will precisely be the locations where
%% $G$ and $\FG$ differs (see theorem \ref{comp-fg-g}). The first of these 3-odd numbers is
%% $7 = F_3+F_5$, followed by $15 = F_3 + F_7$ and $20 = F_3+F_5+F_7$.

%% A few properties of 3-even and 3-odd numbers:
%% \begin{theorem}\label{threeevenodd}
%% \noindent
%% \begin{enumerate}
%% \item A number $n$ is 3-odd if and only if it admits
%% a \emph{lax} decomposition of the form $F_3 + F_{2p+1}+...$.
%% \item A number $n$ is 3-odd if and only if $low(n)=3$ and
%% $low(n-2)$ is odd.
%% \item A number $n$ is 3-even if and only if $low(n)=3$ and
%% $low(n-2)$ is even.
%% \item When $low(n)$ is even and at least 6, then $n-1$ is 3-odd.
%% \item Two consecutive 3-odd numbers are always apart by 5 or 8.
%% \end{enumerate}
%% \end{theorem}

%% \begin{proof}
%% \noindent
%% \begin{enumerate}
%% \item
%%   The left-to-right implication is obvious, since a canonical
%%   decomposition can in particular be seen as a lax one. Suppose
%%   now the existence of such a lax decomposition. During its
%%   normalization, the $F_3$ will necessarily be left intact, and the
%%   term $F_{2p+1}$ can grow, but only by steps of 2.
%% \item Once again, the left-to-right implication is
%%   obvious. Conversely, we start with the canonical decomposition of
%%   $n-2=F_{2p+1}+\fibrest$. If $p>1$, this leads to the desired canonical
%%   decomposition of $n$ as $F_2+F_{2p+1}+\fibrest$. And the case $p\le 1$ is
%%   impossible: $p=0$ leads to $F_1$ being part of a lax
%%   decomposition, and if we assume $p=1$, then we could turn the
%%   decomposition of $n-2$ into a canonical decomposition of $n-4$
%%   whose lowest rank is at least 5. This gives us a lax
%%   decomposition of $n$ of the form $F_2+F_4+\fibrest$. After normalization,
%%   we would obtain that $low(n)$ is even, which is contradictory with
%%   $low(n)=3$.
%% \item We proceed similarly for $low(n)=3$ and $low(n-2)$ even implies
%%   that $n$ is 3-even. First we have a canonical decomposition of
%%   $n-2 = F_{2p} + \fibrest$. Then $p=0$ is illegal, and $p=1$
%%   would give a lax
%%   decomposition of $n$ starting by 3, hence $low(n)$ even, impossible.
%%   And $p>1$ allows to write $n = F_3+F_{2p}+\fibrest$ in a canonical way.
%% \item When $n = F_{2p}+\fibrest$ with $p\ge 3$, then
%%   $n-1 = F_3+F_5+...+F_{2p-1}+\fibrest$
%%   and the condition on $p$ ensures that the terms $F_3$ and $F_5$ are
%%   indeed present.
%% \item If $n$ is 3-odd, it could be written $F_3+F_{2p+1}+\fibrest$.
%%   When $p>2$, then $n+5 = F_3+F_5+F_{2p+1}+\fibrest$ is also 3-odd.
%%   When $p=2$, then $n+8$ is also 3-odd. For justifying that,
%%   we need to consider the third term in the 
%%   canonical decomposition (if any).
%%   \begin{itemize}
%%   \item Either $n$ is of the form $F_3+F_5+F_7+...$
%%     Then $n+F_6 = F_3 + F_5 + F_8 + ...$ is a lax
%%     decomposition of $n+8$, hence $n+8$ is 3-odd (see the first
%%     part of the current theorem above).
%%   \item Either $n$ is of the form $F_3+F_5+\fibrest$
%%     where all terms in $\fibrest$ are strictly greater than $F_7$.
%%     Then $n+F_6 = F_3 + F_7 + \fibrest$ is a lax
%%     decomposition of $n+8$, hence $n+8$ is 3-odd.
%%   \end{itemize}
%%   We finish the proof by considering all intermediate numbers between
%%   $n$ and $n+8$ and we show that $n+5$ is the only one of them that
%%   might be 3-odd:
%%   \begin{itemize}
%%   \item $low(n+1)$ is even (cf. theorem \ref{fibsucc}).
%%   \item $n+2 = F_2+F_4+F_{2p+1}+\fibrest$, and normalizing this
%%     lax decomposition shows that $low(n+2)=2$.
%%   \item $n+3 = F_3+F_4+F_{2p+1}+\fibrest$, and normalizing this
%%     lax decomposition will either combine $F_4$ with some
%%     higher terms (leading to an odd second-lowest term and
%%     hence $n+3$ is 3-even) or combine $F_4$ with $F_3$ (in which
%%     case $low(n+3)\ge 5$).
%%   \item $n+4 = F_2+F_3+F_4+F_{2p+1}+\fibrest$, hence $low(n+4)$ is
%%     even.
%%   \item $n+6 = F_6+F_{2p+1}+\fibrest$ is a lax decomposition
%%    whose lowest rank is either 6 (when $p>2$) or 5 (when $p=2$).
%%    After normalization, we obtain $low(n+6)\ge 5$.
%%   \item Hence $low(n+7)=2$ by last case of theorem \ref{fibsucc}.
%%   \end{itemize}
%% \end{enumerate}
%% \end{proof}

\section{The $H$ function}

This section corresponds to file \doc{GenG}.

\subsection{Definition and initial study of $H$}

\begin{theorem}\label{defG}There exists a unique function
  $H:\mathbb{N}\to\mathbb{N}$ which satisfies the following
  equations:
  $$H(0) = 0$$
  $$\forall n>0,~~ H(n) = n - H(H(H(n-1)))$$
\end{theorem}
\begin{proof}
We need to scrutinize the recursive calls of $H$ on $H(n-1)$ and then
on $H(H(n-1))$, and ensure progressively that these recursive calls
are done on legitimate values. Fortunately $H(n-1)$ and
$H(H(n-1))$ will always be non-negative and strictly lower than $n$, so
calling $H$ on them does indeed have a meaning.
More formally, we prove by induction on $n$ the following
statement: for all $n$, there exists a sequence $H_0...H_n$
of numbers in $[0..n]$ such that $H_0=0$ and
$\forall k\in[1..n], H_k = k - H_{v}$ where $v = H_u$ and $u = H_{k-1}$.
\begin{itemize}
\item For $n=0$, an adequate sequence is obviously $H_0 = 0$.
\item For $n>0$, suppose we have already proved the existence
  of an adequate sequence $H_0...H_n$.
  In particular $H_n \in [0..n]$, let's call it $u$.
  Then $H_u$ is well-defined and also in $[0..n]$, let's call it $v$.
  Then $H_v$ is also well-defined and also in $[0..n]$.
  Finally $(n+1)-H_v$ is in $[1..(n+1)]$. We define $H_{n+1}$ to be
  equal to this $(n+1)-H_v$ and obviously keep the previous values of the
  sequence. All these values are indeed in $[0..(n+1)]$, and the
  recursive equations are satisfied up to $k=n+1$.
\end{itemize}
All these finite sequences that extend each other
lead to an infinite sequence $(H_n)_{n\in\mathbb{N}}$ of natural
numbers, which can also be seen as a function
$H:\mathbb{N}\to\mathbb{N}$, that satisfy the desired equations
by construction.

For the uniqueness, we should first prove that any function $f$
satisfying $f(0)=0$ and our recursive equation above is such that
$\forall n, 0\le f(n)\le n$. This proof can be done by strong
induction over $n$, and a bit of upper and lower bound manipulation.
Then we could prove (still via strong induction
over $n$) that $\forall n, f(n)=g(n)$ when $f$ and $g$
are any functions satisfying our equations.
\end{proof}

The initial values of $H$ are
$H(0)=0$, $H(1)=H(2)=1$, $H(3)=2$, $H(4)=3$, $H(5)=H(6)=4$.

We then establish some basic properties of $H$. Actually these initial
properties are also verified by Hofstadter's function $G$
and all other functions built on the same model (with any number of
nested recursive calls).
\begin{theorem}\label{Gprops}
\noindent
\begin{enumerate}
\item $\forall n, 0 \le H(n) \le n$.
\item $\forall n>0, H(n)=H(n-1)$ implies $H(n+1)=H(n)+1$.
\item $\forall n, H(n+1)-H(n) \in \{0,1\}$.
\item $\forall n\,m, n\le m$ implies $0 \le H(m)-H(n) \le m-n$.
\item $\forall n, H(n)=0$ if and only if $n=0$.
\item $\forall n>1, H(n)<n$.
\item $\forall n, H(2n)\ge n$.
\end{enumerate}
\end{theorem}
\begin{proof}
\noindent
\begin{enumerate}
\item Already seen during the definition of $H$.
\item $H(n+1)-H(n) = (n+1)-H(H(H(n)))-n+H(H(H(n-1)))$.
If $H(n)=H(n-1)$ then the previous expression simplifies to 1.
\item By strong induction over $n$. First, $H(1)-H(0)=1-0$. Then
for a given $n>0$, we assume $\forall k<n, H(k+1)-H(k) \in \{0,1\}$.
By induction hypothesis for $k=n-1$, $H(n)-H(n-1) \in\{0,1\}$.
If it is 0, then $H(n+1)-H(n) = 1$ as before. If it is 1, we could use another
induction hypothesis for $k=H(n-1)$ (and hence $k+1 = H(n)$), leading
to $H(H(n))-H(H(n-1)) \in \{0,1\}$. If this is 0, then once again
$H(n+1)-H(n) = 1$. If it is 1, we use a last induction hypothesis for
$k=H(H(n-1))$ (and hence $k+1 = H(H(n))$), leading to
$H(H(H(n)))-H(H(H(n-1))) \in \{0,1\}$ and hence also $H(n+1)-H(n) \in \{0,1\}$.
\item Mere iteration of the previous result between $n$ and $m$.
\item Point (4) used between 1 and $n$ gives :
  $\forall n\ge 1$, $0 \le H(n)-H(1)$ and $H(1)=1$.
\item Point (4) used between 2 and $n$ gives :
  $\forall n\ge 2$, $H(n)-H(2) \le n-2$ and $H(2)=1$.
\item Points (2) and (3) show that H grow by at least 1 every 2 steps,
  since any stagnation is followed by at least a growth.
\end{enumerate}
\end{proof}

The previous facts shows that $lim_{+\infty} H = +\infty$.
Taking into account this limit and $H(0)=0$ and the growth by steps of
1, we can also deduce that $H$ is onto: any natural number has at
least one antecedent by $H$. Moreover there cannot exists more than
two antecedents for a given value: by monotonicity, these antecedents
are neighbors,
and having more than two would contradict the ``stagnation followed
by growth'' rule.
We can even provide explicitly one of these antecedent:

\TODO TO BE CONTINUED...

%% \begin{theorem}\label{Honto}
%% $\forall n, H(n+H(n))=n$.
%% \end{theorem}
%% \begin{proof}
%% We've just proved that $n$ has at least one antecedent, and
%% no more than two. Let $k$ be the largest of these antecedents.
%% Hence $G(k)=n$ and $G(k+1)\neq n$, leading to $G(k+1)=G(k)+1$.
%% If we re-inject this into the defining equation $G(k+1) = k+1 -
%% G(G(k))$, we obtain that $G(k)+G(G(k))=k$ hence $n+G(n)=k$,
%% and finally $G(n+G(n))=G(k)=n$.
%% \end{proof}

%% As shown during the previous proof, $n+H(n)$ is actually the largest antecedent
%% of $n$ by $H$. In particular $H(n+H(n)+1)=n+1$. And if $n$ has another
%% antecedent, it will hence be $n+H(n)-1$.

%% From this, we can deduce a first relationship between $G$ and
%% Fibonacci numbers.
%% \begin{theorem}\label{Gfib} For all $k\ge 2$, $G(F_k)=F_{k-1}$.
%% \end{theorem}
%% \begin{proof}By induction over $k$. First, $G(F_2)=G(1)=1=F_1$.
%% Take now a $k\ge 2$ and assume that $G(F_k)=F_{k-1}$.
%% Then $G(F_{k+1})=G(F_k+F_{k-1})=G(F_k+G(F_k))=F_k$.
%% \end{proof}

%% Moreover, for $k>2$, $F_k = F_{k-1}+G(F_{k-1})$ is the largest antecedent of
%% $F_{k-1}$, hence $G(1+F_k)=1+F_{k-1}$.

%% We could also establish a alternative equation for $G$, which will be
%% used during the study of function $\FG$:

%% \begin{theorem}\label{Galt} For all $n$ we have $G(n) + G(G(n+1)-1) = n$.
%% \end{theorem}
%% \begin{proof}
%% First, this equation holds when $n=0$ since
%% $G(0)+G(G(1)-1) = G(0) + G(1-1) = 0$.
%% We consider now a number $n\neq 0$. Either $G(n+1)=G(n)$ or $G(n+1)=G(n)+1$.
%% \begin{itemize}
%% \item In the first case, $G(n-1)$ cannot be equal to $G(n)$ as well,
%% otherwise $G$ would stay flat longer than possible. Hence $G(n-1)=G(n)-1$.
%% Hence $G(n)+G(G(n+1)-1) = G(n) + G(G(n)-1) = G(n)+G(G(n-1)) = n$.
%% \item In the second case:
%% $G(n) + G(G(n+1)-1) = G(n+1)-1 + G(G(n)) = (n+1)-1 = n$.
%% \end{itemize}
%% \end{proof}

%% \subsection{The associated $G$ tree}

%% Hofstadter proposed to associate the $G$ function with an infinite
%% tree satisfying the following properties:
%% \begin{itemize}
%% \item The nodes are labeled by all the positive numbers,
%%  in the order of a left-to-right breadth-first traversal,
%%  starting at 1 for the root of the tree.
%% \item For $n\neq 0$, the node $n$ is a child of the node $g(n)$.
%% \end{itemize}

%% In practice, the tree can be constructed progressively, node after node.
%% For instance, the node $2$ is the only child of $1$, and $3$
%% is the only child of $2$, while $4$ and $5$ are the children of $3$.
%% Now come the nodes $6$ and $7$ on top of $4$, and so on. The picture
%% below represents the tree up to depth 7 (assuming that the root is
%% at depth 0).

%% \bigskip

%% \begin{tikzpicture}[grow'=up]
%% \Tree
%%  [.1 [.2 [.3
%%        [.4 [.6 [.9 [.14 22 23 ] [.15 24 ] ]
%%                [.10 [.16 25 26 ]]]
%%            [.7 [.11 [.17 27 28 ] [.18 29 ]]]]
%%        [.5 [.8 [.12 [.19 30 31 ] [.20 32 ]]
%%                [.13 [.21 33 34 ]]]]]]]
%% \end{tikzpicture}

%% This construction process will always be successful : when adding the
%% new node $n$, this new node $n$ will be linked
%% to a parent node already constructed, since $G(n)<n$ as soon as $n>1$.
%% Moreover, $G$ is monotone and
%% grows by at most 1 at each step, hence the position of the new node
%% $n$ will be compatible with the left-to-right breadth-first ordering:
%% $n$ has either the same parent as $n-1$, and we place $n$ to the right
%% of $n-1$, or $G(n)=G(n-1)+1$ in which case $n$ is either to be placed
%% on the right of $n-1$, or at a greater depth.

%% Moreover, since $G$ is onto, each node will have at least one child,
%% and we have already seen that each number has at most two antecedents
%% by $G$, hence the node arities are 1 or 2.

%% \subsubsection*{Tree depth}
%% On the previous picture, we can notice that the rightmost nodes are
%% Fibonacci numbers, while the leftmost nodes have the form $1+F_k$.
%% Before proving this fact, let us first define more properly a depth
%% function.

%% \begin{definition}
%% For a number $n>0$, we note $\depth(n)$ the least number $d$ such that
%% $G^d(n)$ (the $d$-th iteration of $G$ on $n$) reaches 1.
%% We complete this definition by choosing arbitrarily $\depth(0)=0$.
%% \end{definition}

%% For $n>0$, such a number $d$ is guaranteed to exists since $G(k)<k$ as long as
%% $k$ is still not $1$, hence the sequence $G^k(n)$ is strictly
%% decreasing as long as it hasn't reached 1.
%% In particular, we have $\depth(1)=0$, and for all $n>1$ we have
%% $\depth(n) = 1+\depth(G(n))$. Hence our depth function is compatible
%% with the usual notion of depth in a tree.

%% \begin{theorem}\label{depthprops}
%% \noindent
%% \begin{enumerate}
%% \item For all $n$, $\depth(n)=0$ if and only if $n\le 1$.
%% \item For $n>1$ and $k<\depth(n)$, we have $G^k(n) > 1$.
%% \item For $n,m>0$, $n\le m$ implies $\depth(n)\le \depth(m)$.
%% \item For $k>1$, $\depth(F_k) = k-2$.
%% \item For $k>1$, $\depth(1+F_k) = k-1$.
%% \item For $k>0,n>0$, $\depth(n)=k$ if and only if $1+F_{k+1} \leq n \leq
%%   F_{k+2}$.
%% \end{enumerate}
%% \end{theorem}
%% \begin{proof}
%% \noindent
%% \begin{enumerate}
%% \item The statement is valid when $n=0$. For a $n>0$, 
%% $\depth(n)=0$ means $G^0(n)=1$ i.e. $n=1$.
%% \item Consequence of the minimality of depth.
%% \item Consequence of the monotony of $G$ :
%%   for all $k$, $G^k(n)\le G^k(m)$ hence the iterates of $n$ will reach
%%   1 faster than the iterates of $m$.
%% \item $G$ maps a Fibonacci number to the previous one, $k-2$
%%   iterations of $G$ on $F_k$ leads to $F_2=1$.
%% \item $G$ maps a successor of a Fibonacci number to the previous such
%%   number, $k-2$ iterations of $G$ on $1+F_k$ leads to $1+F_2 = 2$.
%% \item If $1+F_{k+1} \leq n \leq F_{k+2}$, the monotony of depth and the
%% previous facts gives $k \le \depth(n) \le k$, hence $\depth(n)=k$.
%% Conversely, when $\depth(n)=k$, we cannot have $n < 1 + F_{k+1}$ otherwise
%% $n \le F_{k+1}$ and hence $\depth(n)\le k-1$, and we cannot have $n > F_{k+2}$
%% otherwise $n \ge 1+F_{k+2}$ and hence $\depth(n)\ge k+1$.
%% \end{enumerate}
%% \end{proof}

%% The previous characterization of $\depth(n)$ via Fibonacci bounds shows
%% that depth could also have been defined thanks to {\tt
%%   fib\_inv$(n-1)-1$}, see \ref{fibinv}. It also shows that the number of
%% nodes at depth $k$ is $F_{k+2}-(1+F_{k+1})+1 = F_{k}$ as soon as
%% $k\neq 0$.


%% \subsubsection*{The shape of the $G$ tree}

%% If we ignore the node labels and concentrate on the shape of the $G$
%% tree, we encounter a great regularity. $G$ starts with two unary
%% nodes that are particular cases (labeled 1 and 2), and after that
%% we encounter a sub-tree $G'$ whose shape is obtained by repetitions
%% of the same basic pattern:

%% \bigskip

%% G = 
%% \begin{tikzpicture}[grow'=up]
%% \Tree [.$\bullet$ [.$\bullet$ G' ]]
%% \end{tikzpicture}
%% ~~and~~
%% G' =
%% \begin{tikzpicture}[grow'=up]
%% \Tree [.$\bullet$ [.G' ] [.$\bullet$ [.G' ] ]]
%% \end{tikzpicture}

%% G' has hence a fractal shape: it appears as a sub-tree of itself. To
%% prove the existence of such a pattern, we study the arity of children
%% nodes.
%% \begin{theorem}\label{Gnodes}
%% In the $G$ tree, a binary node has a left child which is also binary,
%% and a right child which is unary, while the unique child of a unary
%% node is itself binary.
%% \end{theorem}
%% \begin{proof}
%% First, we show that the leftmost child of a node is always binary.
%% Let $n$ be a node, and $p$ its leftmost child, i.e. $G(p)=n$ and
%% $G(p-1)=n-1$. We already know one child $q=p+G(p)$ of $p$, let's now
%% show that $q-1$ is also a child of $p$. For that, we consider $p-1$
%% and its rightmost child $q'=p-1+G(p-1)$. Rightmost means that
%% $G(q'+1)=p$. So $G(q-1)=G(p+G(p)-1)=G(p+n-1)=G(p+G(p-1))=G(q'+1)=p$
%% and we can conclude.

%% We can now affirm that the unique child of a unary node is itself
%% binary, since this child is in particular the leftmost one.

%% Let's now consider a binary node $n$, with its right child $p=n+G(n)$
%% and its left child $p-1$: $G(p)=G(p-1)=n$.

%% \begin{tikzpicture}[grow'=up]
%% \Tree [.$n$ [.$p-1$ $q-2$ $q-1$ ] [.$p$ $q$ ]]
%% \end{tikzpicture}

%% The leftmost child $p-1$
%% is already known to be binary. Let's now show that the right child $p$
%% is unary. If $p$ has a second child, it will be $q-1$, but:
%% $G(q-1) = G(p+G(p)-1) = G(p-1+G(p-1)) = p-1$.
%% so $q-1$ is a child of $p-1$ rather than $p$, and $p$ is indeed
%% a unary node.
%% \end{proof}

%% \subsection{$G$ and Fibonacci decompositions}

%% \begin{theorem}\label{Gshift}
%% Let $n = \fibrest$ be a lax Fibonacci decomposition.
%% Then $G(n) = \Sigma F_{i-1}$, the Fibonacci decomposition obtained
%% by removing 1 at all the ranks in the initial decomposition.
%% \end{theorem}

%% For instance $G(11) = G(F_4+F_6) = F_3 + F_5 = 7$. Note that the
%% obtained decomposition isn't necessarily lax anymore, since a
%% $F_1$ might have appeared if $F_2$ was part of the initial
%% decomposition. The theorem also holds for canonical decompositions
%% since they are a particular case of lax decompositions. Once again,
%% the obtained decompositions might not be canonical since $F_1$ might
%% appear, but in this case we could turn this $F_1$ into a $F_2$. This
%% gives us a lax decomposition that we can re-normalize. For
%% instance $G(12) = G(F_2+F_4+F_6) = F_1+F_3+F_5 = F_2+F_3+F_5 = F_4+F_5
%% = F_6 = 8$.
 
%% \begin{proof}
%% We proceed by strong induction over $n$. The case $0$ is obvious,
%% since the only possible decomposition is the empty one, and the
%% ``shifted'' decomposition is still empty, as required by $G(0)=0$.
%% We now consider a number $n\neq 0$, with a non-empty lax decomposition
%% $n = F_k+\fibrest$, and assume the statement to be true for all $m<n$.
%% We will use the recursive equation $G(n)=n-G(G(n-1))$, and distinguish
%% many cases according to the value of $k$.
%% \begin{itemize}
%% \item Case $k=2$. Then $n-1 = \fibrest$. A first induction
%%   hypothesis gives $G(n-1) = \Sigma F_{i-1}$. Since the initial
%%   decomposition was lax, $\fibrest$ cannot contain $F_2$,
%%   hence $\Sigma F_{i-1}$ is still a lax decomposition. As seen
%%   many times now, $G(n-1)<n$ here, and we're free to use a second
%%   induction hypothesis leading to $G(G(n-1)) = \Sigma F_{i-2}$.
%%   Finally $G(n) = 1+\Sigma F_i - \Sigma F_{i-2} = 1 + \Sigma (F_i-F_{i-2})
%%    = F_1 + \Sigma F_{i-1}$, as required.

%% \item Case $k$ even and distinct from 2. Then $k$ can be written $2p$ with $p>1$.
%%   Then $n-1 = F_{2p}-1 + \Sigma F_i = F_3+F_5+...+F_{2p-1}+\Sigma F_i$.
%%   The induction hypothesis on $n-1$ gives:
%%   $$G(n-1) = F_2+F_4+...+F_{2p-2}+\Sigma F_{i-1}$$
%%   Since this is still a lax decomposition, we use a second
%%   induction hypothesis on it, hence:
%%   $$G(G(n-1)) = F_1+F_3+...+F_{2p-3}+\Sigma F_{i-2}$$
%%   $$G(G(n-1)) = 1 + (F_{2p-2}-1)+\Sigma F_{i-2}$$
%%   And so:
%%   $$G(n) = F_{2p}-F_{2p-2} + \Sigma (F_i - F_{i-2}) =
%%            F_{2p-1} + \Sigma F_{i-1}$$

%% \item Case $k$ odd. Then $k$ can be written $2p+1$ with $p>0$.
%%   Then $n-1 = F_{2p+1}-1 + \Sigma F_i = F_2+F_4+...+F_{2p}+\Sigma F_i$.
%%   The induction hypothesis on $n-1$ gives:
%%   $$G(n-1) = F_1+F_3+...+F_{2p-1}+\Sigma F_{i-1}$$
%%   We turn the $F_1$ above into a $F_2$, obtaining again a lax
%%   decomposition, for which a second induction hypothesis gives:
%%   $$G(G(n-1)) = F_1+F_2+...+F_{2p-2}+\Sigma F_{i-2}$$
%%   $$G(G(n-1)) = 1 + (F_{2p-1}-1)+\Sigma F_{i-2}$$
%%   And so:
%%   $$G(n) = F_{2p+1}-F_{2p-1} + \Sigma (F_i - F_{i-2}) =
%%            F_{2p} + \Sigma F_{i-1}$$
%% \end{itemize}
%% \end{proof}

%% Thanks to this characterization of the effect of $G$ on Fibonacci
%% decomposition, we can now derive a few interesting properties.

%% \begin{theorem}\label{Gclass1}
%% \noindent
%% \begin{enumerate}
%% \item $G(n+1)=G(n)$ if and only if $low(n)=2$.
%% \item If $low(n)$ is odd, then $G(n-1)=G(n)$.
%% \item If $low(n)$ is even, then $G(n-1)=G(n)-1$.
%% \item For $n\neq 0$, the node $n$ of tree $G$
%%  is unary if and only if $low(n)$ is odd.
%% \end{enumerate}
%% \end{theorem}
%% \begin{proof}
%% \noindent
%% \begin{enumerate}
%% \item
%% If $low(n)=2$, then we can write a canonical decomposition
%% $n=F_2+\fibrest$, hence
%% $n+1=F_3+\fibrest$ is a lax decomposition. By the previous
%% theorem, $G(n) = F_1 + \Sigma F_{i-1}$ while
%% $G(n+1) = F_2 + \Sigma F_{i-1}$, hence the desired equality.
%% Conversely, we consider a canonical decomposition of $n$ as $\fibrest$.
%% If $F_2$ isn't part of this decomposition, then $n+1=F_2+\fibrest$
%% is a correct lax decomposition, leading to
%% $G(n+1)=1+\Sigma F_{i-1}=1+G(n)$. So $low(n)\neq 2$ implies
%% $G(n+1)\neq G(n)$.

%% \item If $low(n)$ is odd, we've already shown in theorem \ref{fibpred}
%%   that $low(n-1)=2$ hence $G(n)=G(n-1)$ by the previous point.

%% \item If $low(n)$ is even, the same theorem
%%  \ref{fibpred} shows that $low(n-1)$ is 3 or more, provided that $n>1$.
%%  In this case the first point above allows to conclude. We now check separately
%%  the cases $n=0$ (irrelevant here since $low(0)$ doesn't exists) and
%%  $n=1$ (for which $G(1-1)$ is indeed $G(1)-1$).

%% \item We've already seen that a node $n$ is binary whenever 
%% $G(n+G(n)-1) = n$. When $low(n)$ is odd, we have
%% $G(n-1)=G(n)$, hence $G(n+G(n)-1) = G(n-1+G(n-1)) = n-1$, hence $n$ is
%%   unary. When $low(n)$ is even, we have
%% $n+G(n)-1 = (n-1+G(n-1))+1$ : this is the next node to the right after
%% the rightmost child of $n-1$, its image by $G$ is hence $n$, and
%% finally $n$ is indeed binary.
%% \end{enumerate}
%% \end{proof}


%% \begin{theorem}\label{Glow}
%% If $low(n)>2$, then $low(G(n))=low(n)-1$. In particular,
%% if $low(n)$ is odd, then $low(G(n))$ is even, and if $low(n)$
%% is even and different from 2, then $low(G(n))$ is odd.
%% \end{theorem}
%% \begin{proof}
%% This is a direct application of theorem \ref{Gshift}:
%% when $low(n)>2$, the decomposition we obtain for $G(n)$ is
%% still canonical, and its lowest rank is $low(n)-1$.
%% The statements about parity are immediate consequences.
%% \end{proof}

%% \begin{theorem}\label{Gclass2}
%% \noindent
%% \begin{enumerate}
%% \item If $low(n)=2$, then $low(G(n))$ is even.
%% \item If $low(n)=3$, then $low(G(n))=2$.
%% \item If $low(n)>3$, then $low(G(n))>2$ and $low(G(n)+1)$ is even.
%% \end{enumerate}
%% \end{theorem}
%% \begin{proof}
%% \noindent
%% \begin{enumerate}
%% \item Once again, we consider a canonical decomposition
%% $n = F_2 + \fibrest$. Then $G(n) = F_1 + \Sigma F_{i-1}$. If we
%% turn the $F_1$ into a $F_2$, we obtain a lax decomposition,
%% that can be normalized into a canonical decomposition whose
%% lowest rank will hence be even.
%% \item Direct application of previous theorem: $low(G(n))=low(n)-1$.
%% \item Here also, $low(G(n))=low(n)-1$, hence $low(G(n))$ cannot
%% be 2 here. Then the theorem \ref{fibsucc} implies that $low(G(n)+1)$ is even.
%% \end{enumerate}
%% \end{proof}

%% \begin{theorem}\label{Gthree}
%% \noindent
%% \begin{enumerate}
%% \item If $n$ is 3-even, then $G(n)+1$ is 3-odd.
%% \item If $n$ is 3-odd, then either $G(n)+1$ is 3-even or $low(G(n)+1)>3$.
%% \end{enumerate}
%% \end{theorem}
%% \begin{proof}
%% \noindent
%% \begin{enumerate}
%% \item Take a canonical decomposition $n=F_3+F_{2p}+\fibrest$.
%% Hence $G(n)+1=1+F_2+F_{2p-1}+\Sigma F_{i-1}=F_3+F_{2p-1}+\Sigma F_{i-1}$,
%% and this decomposition is still canonical (for canonicity reasons,
%% $p>2$).
%% \item Similarly, $n=F_3+F_{2p+1}+\fibrest$ implies
%% $G(n)+1=F_3+F_{2p}+\Sigma F_{i-1}$. When $p>1$, this decomposition
%% is canonical and $G(n)+1$ is 3-even. When $p=1$, this decomposition
%% is only a lax one, starting by $F_3+F_4+...$. Its normalization
%% will hence end with a lowest rank of at least 5.
%% \end{enumerate}
%% \end{proof}

%% \subsection{$G$ and its ``derivative'' $\Delta G$}
%% \label{deltaG}

%% We consider now the ``derivative'' $\Delta G$ of $G$, defined via
%% $\Delta G(n) = G(n+1)-G(n)$. We already know from theorem
%% \ref{Gprops} that the output of $\Delta G$ is always either 0 or 1.

%% \begin{theorem}\label{Gdelta}
%% For all $n$, $\Delta G(n+1) = 1 - \Delta G(n).\Delta G(G(n))$.
%% \end{theorem}
%% \begin{proof}
%% We already know that $\Delta G(n)=0$ implies $\Delta G(n+1)=1$:
%% we cannot have $G(n)=G(n+1)=G(n+2)$.
%% Consider now a $n$ such that
%% $G(n+1)-G(n)=1$. By using the recursive definition of $G$, we have
%% $G(n+2)-G(n+1)=(n+2-G(G(n+1)))-(n+1-G(G(n))) = 1 - (G(G(n+1))-G(G(n))$.
%% Hence $\Delta G(n+1) = 1 - (G(G(n)+1)-G(G(n)) = 1 - \Delta G(G(n))$.
%% \end{proof}

%% This equation provides a way to express $G(n+2)$ in terms of
%% $G(n+1)$ and $G(n)$ and $G(G(n))$ and $G(G(n)+1)$. Since
%% $n+1$, $n$, $G(n)$ and $G(n)+1$ are all strictly less than $n+2$,
%% we could use this equation as an alternative way to define
%% recursively $G$,
%% alongside two initial equations $G(0)=0$ and $G(1)=1$.
%% We proved in Coq that $G$ is indeed the unique function to
%% satisfy these equations (see {\tt GD\_unique} and
%% {\tt  g\_implements\_GD}).

%% \section{The $\FG$ function}

%% This section corresponds to file \doc{FlipG}.
%% Here is a quote from page 137 of Hofstadter's book \cite{GEB}:
%% \begin{quote}
%% A problem for curious readers is: suppose you flip Diagram G
%% around as if in a mirror, and label the nodes of the new tree so they
%% increase from left to right. Can you find a recursive \emph{algebraic}
%% definition for this ``flip-tree'' ?
%% \end{quote}

%% \subsection{The $\flip$ function}

%% Flipping the $G$ tree as if in a mirror is equivalent to keeping
%% its shape unchanged, but labeling the nodes from right to left
%% during the breadth-first traversal. Let us call $\flip(n)$ the
%% new label of node $n$ after this transformation.
%% We've seen in the previous section that given a depth $k\neq 0$,
%% the nodes at this depth are labeled from $1+F_{k+1}$ till $F_{k+2}$.
%% After the $\flip$ transformation, the $1+F_{k+1}$ node and
%% the $F_{k+2}$ node will hence have exchanged their label.
%% Similarly, $2+F_{k+1}$ will become $F_{k+2}-1$ and vice-versa.
%% More generally, if $\depth(n)=k$, the distance between
%% $\flip(n)$ and the leftmost node $1+F_{k+1}$ will be equal to
%% the distance between $n$ and the rightmost node $F_{k+2}$,
%% hence:
%% $$\flip(n) - (1+F_{k+1}) = F_{k+2} - n$$
%% So:
%% $$\flip(n) = 1+F_{k+3}-n$$
%% We finally complete this definition to handle the case $n\le 1$:
%% \begin{definition}
%% We define the function $\flip : \mathbb{N}\to\mathbb{N}$
%% in the following way:
%% $$\flip(n) = if~(n\le 1)~then~n~else~1+F_{3+\depth(n)}-n$$
%% \end{definition}

%% A few properties of this $\flip$ function:
%% \begin{theorem}\label{flipprops}
%% \noindent
%% \begin{enumerate}
%% \item For all $n\in\mathbb{N}$, $\depth(\flip(n)) = \depth(n)$.
%% \item For all $n$, $n>1$ if and only if $\flip(n)>1$.
%% \item For $1 \le n \le F_{k}$, we have
%%  $\flip(F_{k+1}+n) = 1+F_{k+2}-n$.
%% \item $\flip$ is involutive.
%% \item For all $n>1$, if $\depth(n+1)=\depth(n)$ then
%%   $\flip(n+1)=\flip(n)-1$.
%% \item For all $n>1$, if $\depth(n-1)=\depth(n)$ then
%%   $\flip(n-1)=\flip(n)+1$.
%% \end{enumerate}
%% \end{theorem}
%% \begin{proof}
%% \noindent
%% \begin{enumerate}
%% \item If $n\le 1$, then $\flip(n)=n$ and the property is obvious.
%% For $n>1$, if we name $k$ the depth of $n$, we have
%% $\flip(n) = 1+F_{k+3}-n$. Since $1+F_{k+1} \le n \le F_{k+2}$,
%% we hence have $1+F_{k+3}-F_{k+2} \le \flip(n) \le 1+F_{k+3}-1-F_{k+1}$.
%% And finally $1+F_{k+1} \le \flip(n) \le F_{k+2}$, and this characterizes
%% the nodes at depth $k$, hence $\depth(\flip(n))=k=\depth(n)$.
%% \item We know that $n$ and $\flip(n)$ have the same depth, and
%% we've already seen that being less or equal to 1 is equivalent
%% to having depth 0.
%% \item Consider $1 \le n \le F_{k}$. Hence $k$ is at least 1,
%% otherwise $1 \le F_0 = 0$. We have
%% $1+F_{k+1}\le F_{k+1}+n \le F_{k+1}+F_{k} = F_{k+2}$, so the depth
%% of $F_{k+1}+n$ is $k$
%% (still via the same characterization of nodes at depth $k$).
%% Moreover $F_{k+1}+n$ is more than 2,
%% so the definition of $\flip$ gives:
%% $\flip(F_{k+1}+n)=1+F_{k+3}-F_{k+1}-n = 1+F_{k+2}-n$.
%% \item If $n\le 1$ then $\flip(n)=n$ is still less or equal to 1,
%% so $\flip(\flip(n))=\flip(n)=n$.
%% Consider now $n>1$. We already know that $\flip(n)>1$ and
%% $\flip(n)$ and $n$ have same depth (let us name it $k$). Hence:
%% $\flip(\flip(n)) = 1+F_{k+2}-\flip(n)=1+F_{k+2}-(1+F_{2+k}-n) = n$.
%% \item Let us name $k$ the common depth of $n+1$ and $n$.
%% In these conditions
%% $\flip(n+1) = 1+F_{k+2}-(n+1) = (1+F_{k+2}-n)-1 = \flip(n)-1$.
%% \item Similar proof.
%% \end{enumerate}
%% \end{proof}

%% In particular, the third point above shows that for $k>1$
%% we indeed have $\flip(1+F_k) = F_{k+1}$ (and vice-versa since $\flip$
%% is involutive).

%% \subsection{Definition of $\FG$ and initial properties}

%% We can now take advantage of this $\flip$ function to obtain
%% a first definition of the $\FG$ function, corresponding to the
%% flipped $G$ tree:

%% \begin{definition}
%% The function $\FG : \mathbb{N}\to\mathbb{N}$ is defined via:
%% $$\FG(n) = \flip(G(\flip(n))$$
%% \end{definition}

%% We benefit from the involutive aspect of $\flip$ to switch
%% from right-left to left-right diagrams, use $G$, and then
%% switch back to right-left diagram. The corresponding Coq function
%% is named {\tt fg}.

%% By following this definition, the initial values of $\FG$ are
%% $\FG(0)=0$, $\FG(1)=\FG(2)=1$, $\FG(3)=2$. The first difference
%% between $\FG$ and $G$ appears for $\FG(7)=5=G(7)+1$. We'll dedicate a
%% whole section later on the comparison between $\FG$ and $G$.
%% Here is the initial levels of this flipped tree $\FG$,
%% the boxed values being the places where $\FG$ and $G$ differ.

%% \bigskip

%% \begin{tikzpicture}[grow'=up]
%% \Tree
%%  [.1 [.2 [.3
%%        [.4 [.6 [.9 [.14 22 23 ] ]
%%                [.10 [.\fbox{15} 24 ] 
%%                     [.16 25 26 ]]]]
%%        [.5 [.\fbox{7} [.11 [.17 27 ] [.18 \fbox{28} 29 ]]]
%%            [.8 [.12 [.19 30 31 ] ]
%%                [.13 [.\fbox{20} 32 ]
%%                     [.21 33 34 ]]]]]]]
%% \end{tikzpicture}

%% We now show that $\FG$ enjoys the same basic properties as $G$:

%% \begin{theorem}\label{FGprops}
%% \noindent
%% \begin{enumerate}
%% \item For all $n>1$, $\depth(\FG(n)) = \depth(n)-1$.
%% \item For all $k>0$, $\FG(F_{k+1}) = F_k$.
%% \item For all $k>1$, $\FG(1+F_{k+1}) = 1+F_k$.
%% \item For all $n$, $\FG(n+1)-\FG(n) \in \{0,1\}$.
%% \item For all $n,m$, $n\le m$ implies $0 \le \FG(m)-\FG(n) \le m-n$.
%% \item For all $n$, $0 \le \FG(n) \le n$.
%% \item For all $n$, $\FG(n)=0$ if and only if $n=0$
%% \item For all $n>1$, $\FG(n)<n$.
%% \item For all $n\neq 0, \FG(n)=\FG(n-1)$ implies $\FG(n+1)=\FG(n)+1$.
%% \item $\FG$ is onto.
%% \end{enumerate}
%% \end{theorem}
%% \begin{proof}
%% \noindent
%% \begin{enumerate}
%% \item Since $\flip$ doesn't modify the depth, we reuse the
%% result about the depth of $G(n)$.
%% \item $\FG(F_{k+1}) = \flip(G(\flip(F_{k+1}))) = \flip(G(1+F_k)) =
%%        \flip(1+F_{k-1}) = F_k$.
%% \item Similar proof.
%% \item The property is true for $n=0$ and $n=1$.
%% Consider now $n>1$. Let $k$ be $\depth(n)$, which is
%% hence different from 0. We have $1+F_{k+1} \le n \le F_{k+2}$.
%% If $n$ is $F_{k+2}$, we have already seen that
%% $\FG(1+F_{k+2}) = 1+F_{k+1} = 1+\FG(F_{k+2})$. Otherwise $n < F_{k+2}$
%% and $n+1$ has hence the same depth as $n$. Then
%% $\flip(n+1)=\flip(n)-1$. Now, by applying $G$ on the two consecutive values
%% of same depth $\flip(n)$ and $\flip(n)-1$, we obtain two results that
%% are either equal or consecutive, and of same depth. By applying $\flip$
%% again on these $G$ results, we end on two consecutive or equal
%% $\FG$ results.
%% \item Iteration of the previous results between $n$ and $m$.
%% \item Thanks to the previous point, $0 \le \FG(m)-\FG(0) \le m-0$
%% and $\FG(0)=0$.
%% \item $\FG(1)=1$ and $0 \le \FG(m)-\FG(1)$ as soon as $m\ge 1$.
%% \item $\FG(2)=1$ and $\FG(m)-\FG(2) \le m-2$ as soon as $m \ge 2$.
%% \item If $n=1$, then the precondition $\FG(1)=\FG(0)$ isn't
%%   satisfied. If $n=2$ or $n=3$, the conclusion is satisfied:
%%   $\FG(3)=2=\FG(2)+1$ and $\FG(4)=3=\FG(3)+1$.
%%   We now consider $n>3$ such that
%%   $\FG(n)=\FG(n-1)$. We have already seen that $\FG(n+1)-\FG(n)$
%%   is either 0 or 1. If it is 1, we can conclude. We proceed by
%%   contradiction and suppose it is 0. We hence have
%%   $\FG(n-1)=\FG(n)=\FG(n+1)$, or said otherwise
%%   $\flip(G(\flip(n-1))) = \flip(G(\flip(n))) = \flip(G(\flip(n+1)))$.
%%   Since $\flip$ is involutive hence bijective, this implies that
%%   $G(\flip(n-1))=G(\flip(n))=G(\flip(n+1))$. To be able to commute
%%   $\flip$ and the successor/predecessor, we now study the depth
%%   of these values.
%%   \begin{itemize}
%%   \item If $n+1$ has a different depth than $n$, then
%%     $n$ is a Fibonacci number $F_{k}$, with $k>4$ since $n>3$.
%%     Then $\flip(n+1)=F_{k+1}$ and $\flip(n)=1+F_{k-1}$. These values
%%     aren't equal nor consecutive, hence $G$ cannot give the same
%%     result for them, this situation is indeed contradictory.
%%   \item Similarly, if $n-1$ has a different depth than $n$, then
%%     $n-1$ is a Fibonacci number $F_k$, with $k>3$ since $n>3$.
%%     Then $\flip(n-1)=1+F_{k-1}$ while $\flip(n)=F_{k+1}$.
%%     Once again, these values aren't equal nor consecutive,
%%     hence $G$ cannot give the same result for them. Contradiction.
%%   \item In the last remaining case, 
%%     $\depth(n-1)=\depth(n)=\depth(n+1)$.
%%     Then $\flip(n+1)=\flip(n)-1$ and $\flip(n-1)=\flip(n)+1$ and
%%     these value are at distance 2, while having the
%%     same result by $G$, which is contradictory.
%%   \end{itemize}
%% \item We reason as we did earlier for $G$: $\FG$ starts with
%% $\FG(0)=0$, it grows by steps of 0 or 1, and it grows by at
%% least 1 every two steps. Hence its limit is $+\infty$ and it is
%% an onto function.
%% \end{enumerate}
%% \end{proof}

%% \subsection{An recursive algebraic definition for $\FG$}

%% The following result is an answer to Hofstadter's problem.
%% It was already mentioned on OEIS page \cite{OEIS-FG}, but only as
%% a conjecture.

%% \newcommand{\nn}{\overline{n}}

%% \begin{theorem}\label{FGeqn}
%% For all $n>3$ we have $\FG(n) = n+1 - \FG(1+\FG(n-1))$.
%% And $\FG$ is uniquely characterized by this equation
%% plus initial equations $\FG(0)=0$, $\FG(1)=\FG(2)=1$ and
%% $\FG(3)=2$.
%% \end{theorem}
%% \begin{proof}
%% We consider $n>3$. Let $k$ be its depth, which is hence at
%% least 3. We know that $1+F_{k+1} \le n \le F_{k+2}$. We will note
%% $\flip(n)$ below as $\nn$. Roughly speaking, this proof is
%% essentially a use of theorem \ref{Galt} which implies that
%% $G(G(\nn+1)-1) = \nn - G(\nn)$, and then some
%% play with flip and predecessors and successors, when possible,
%% or direct particular proofs otherwise.
%% \begin{itemize}
%% \item We start with the most general case : we suppose here
%% $\depth(n-1)=\depth(n)$ and $\depth(\FG(n-1)+1)=\depth(\FG(n-1)$.
%% Let us shorten $\FG(n-1)+1$ as $p$. In particular
%% $\depth(p)=\depth(\FG(n-1))=\depth(n-1)-1=\depth(n)-1=k-1$.
%% Due to the equalities between depths, and the facts that
%% $n>1$ and $\FG(n-1)>1$, we're allowed to use the
%% last properties of theorem \ref{flipprops} : $\flip(n-1)=\nn+1$ and
%% \begin{align*}
%% \flip(p) &= \flip(\FG(n-1)+1) \\
%%         &= \flip(\FG(n-1))-1 \\
%%         &= \flip(\flip(G(\flip(n-1))))-1 \\
%%         &= G(\flip(n-1))-1 \\
%%         &= G(\nn+1)-1
%% \end{align*}
%% We now exploit the definition of $\flip$ three times:
%% \begin{enumerate}
%% \item $\depth(n)=k\neq 0$ implies $\nn=1+F_{k+3}-n$
%% \item $\depth(G(\flip(p)))=\depth(p)-1=k-2 \neq 0$ hence:
%% $$\FG(p) = \flip(G(\flip(p))) = 1+F_{k+1}-G(\flip(p))$$
%% \item $\depth(G(\nn))=\depth(\nn)-1=\depth(n)-1=k-1 \neq 0$, so:
%% $$\FG(n)=\flip(G(\nn))=1+F_{k+2}-G(\nn)$$
%% \end{enumerate}
%% All in all:
%% \begin{align*}
%% n+1-\FG(1+\FG(n-1)) & = n+1-\FG(p) \\
%%                     & = n+1-(1+F_{k+1}-G(\flip(p))) \\
%%                     & = n-F_{k+1}+G(G(\nn+1)-1)) \\
%%                     & = n-F_{k+1}+(\nn-G(\nn)) \\
%%                     & = n-F_{k+1}+(1+F_{k+3}-n)-G(\nn) \\
%%                     & = 1+F_{k+2}-G(\nn) \\
%%                     & = \FG(n)
%% \end{align*}
%% \item We now consider the case where $\depth(n-1)\neq \depth(n)$.
%% So $n$ is the least number of depth $k$, hence $n=1+F_{k+1}$.
%% In this case:
%% \begin{align*}
%% 1+n-\FG(1+\FG(n-1)) & = 2+F_{k+1}-\FG(1+\FG(F_{k+1})) \\
%%                     & = 2+F_{k+1}-\FG(1+F_{k}) \\
%%                     & = 2+F_{k+1}-(1+F_{k-1}) \\
%%                     & = 1+F_{k} \\
%%                     & = \FG(n)
%% \end{align*}
%% \item The last case to consider is $\depth(n-1)=\depth(n)$ but
%% $\depth(\FG(n-1)+1)\neq \depth(\FG(n-1))$. As earlier, we prove
%% that $\depth(\FG(n-1))=\depth(n-1)-1=\depth(n)-1=k-1$. So
%% $\FG(n-1)$ is the greatest number of depth $k-1$, hence
%% $\FG(n-1)=F_{k+1}$. Since $\FG(F_{k+2})$ is also $F_{k+1}$, then $n-1$
%% and $F_{k+2}$ are at a distance of 0 or 1. But we know that
%% $\depth(n)=k$, hence $n \le F_{k+2}$, so $n-1 < F_{k+2}$.
%% Finally $n=F_{k+2}$ and:
%% \begin{align*}
%% 1+n-\FG(1+\FG(n-1)) & = 1+F_{k+2}-\FG(1+F_{k+1}) \\
%%                     & = 1+F_{k+2}-(1+F_{k}) \\
%%                     & = F_{k+1} \\
%%                     & = \FG(n)
%% \end{align*}
%% \end{itemize}

%% Finally, if we consider another function $F$ satisfying
%% the same recursive equation, as well as the same initial values
%% for $n\le 3$, then we prove by strong induction over $n$ that
%% $\forall n, F(n)=\FG(n)$. This is clear for $n\le 3$. Consider now
%% some $n>3$, and assume that $F(k)=\FG(k)$ for all $k<n$.
%% In particular $F(n-1)=\FG(n-1)$ by induction hypothesis for
%% $n-1<n$.
%% Moreover $n-1>1$ hence $\FG(n-1) < n-1$ hence $\FG(n-1)+1<n$.
%% So we could use a second induction hypothesis at this position:
%% $F(\FG(n-1)+1)=\FG(\FG(n-1)+1)$. This combined with the first
%% induction hypothesis above gives $F(F(n-1)+1)=\FG(\FG(n-1)+1)$
%% and finally $F(n)=\FG(n)$ thanks to the recursive equations
%% for $F$ and $\FG$.
%% \end{proof}

%% \subsection{The $\FG$ tree}

%% Since $\FG$ has been obtained as the mirror of $G$, we already
%% know its shape: it's the mirror of the shape of $G$. We'll
%% nonetheless be slightly more precise here. The proofs given
%% in this section will be deliberately sketchy, please consult
%% theorem {\tt unary\_flip} and alii in file \doclab{FlipG}{unary\_flip}\ for
%% more detailed and rigorous justifications.

%% First, a node
%% $n$ has a child $p$ in the tree $\FG$ is and only if the
%% node $\flip(n)$ has a child $\flip(p)$ in the tree $G$.
%% Indeed, $\FG(p)=n$ iff $\flip(G(\flip(p)))=n$ iff
%% $G(\flip(p))=\flip(n)$. Moreover, a rightmost child in $\FG$
%% corresponds by $\flip$ with a leftmost child in $G$ and vice-versa.
%% Indeed, if $p$ and $n$ aren't on the borders of the trees,
%% then the next node will become the previous by flip, and vice-versa
%% (see theorem \ref{flipprops}). And if $p$ and/or $n$ are on the
%% border, they are Fibonacci numbers or successors of Fibonacci
%% numbers, and we check these cases directly.

%% Similarly, the arity of $n$ in $\FG$ is equal to the arity of
%% $\flip(n)$ in $G$.
%% For instance, if we take a unary node $n$ and its unique
%% child $p$ in $\FG$, this means that $\FG(p+1)=n+1$ and
%% $\FG(p-1)=n-1$. In the most general case $\flip$ will lead
%% to similar properties about $\flip(p)$ and $\flip(n)$ in $G$,
%% hence the fact that $\flip(n)$ is unary in $G$. And we handle
%% particular cases about Fibonacci numbers on the border as usual.

%% \begin{theorem}
%% For all $n>1$, $\flip(\flip(n)+G(\flip(n)))$, which could also be
%% written $\flip(\flip(n)+\flip(\FG(n)))$, is the leftmost child
%% of $n$ in the $\FG$ tree. 
%% \end{theorem}
%% \begin{proof}
%% See the previous paragraph: $\flip(n)+G(\flip(n))$ is rightmost
%% child of $\flip(n)$ in $G$, hence the result about $\FG$.
%% \end{proof}

%% \begin{theorem}
%% For all $n>1$, $n-1+\FG(n+1)$ is the rightmost child of $n$ in the
%% $\FG$ tree.
%% \end{theorem}
%% \begin{proof}
%% We already know that all nodes $n$ in the $\FG$ tree have at
%% least one antecedent, and no more than two.
%% Take $n>1$, and let $k$ be the largest of its antecedents by $\FG$.
%% Hence $\FG(k)=n$ and $\FG(k+1)\neq n$, leading to $\FG(k+1)=n+1$.
%% If we re-inject this into the previous recursive equation
%% $\FG(k+1) = k+2 - \FG(\FG(k)+1)$,
%% we obtain that $n+1 = k+2-\FG(n+1)$ hence $k=n-1+\FG(n+1)$.
%% \end{proof}

%% Of course, for unary nodes, there is only one child, hence the
%% leftmost and rightmost children given above coincide. Otherwise,
%% for binary nodes, they are apart by 1.

%% \begin{theorem}
%% In the $\FG$ tree, a binary node has a right child which is also binary,
%% and a left child which is unary, while the unique child of a unary
%% node is itself binary.
%% \end{theorem}
%% \begin{proof}
%% Flipped version of theorem \ref{Gnodes}.
%% \end{proof}

%% \subsection{Comparison between $\FG$ and $G$}

%% We already noticed earlier that $\FG$ and $G$ produce very similar
%% answers. Let us study this property closely now.

%% \begin{theorem}\label{comp-fg-g}
%% For all $n$, we have $\FG(n)=1+G(n)$ whenever $n$ is 3-odd,
%% and $\FG(n)=G(n)$ otherwise.
%% \end{theorem}
%% \begin{proof}
%% We proceed by strong induction over $n$.
%% When $n\le 3$, $n$ is never \mbox{3-odd}, and we indeed have
%% $\FG(n)=G(n)$. We now consider $n>3$, and assume that
%% the result is true at all positions strictly less than $n$.
%% For comparing $\FG(n)$ and $G(n)$ we use their corresponding
%% recursive equations: when $n$ is 3-odd we need to prove that
%% $\FG(\FG(n-1)+1)=G(G(n-1))$, and otherwise we need to establish
%% that $\FG(\FG(n-1)+1)=1+G(G(n-1))$. So we'll need
%% induction hypotheses (IH) for $n-1<n$ and for $\FG(n-1)+1$
%% (which is indeed strictly less than $n$:
%% since $1<n-1$, we have $\FG(n-1)<n-1$). But the exact equations
%% given by these two IH will depend on the
%% status of $n-1$ and $\FG(n-1)+1$ : are these numbers 3-odd or not ?
%% For determining that, we'll consider the Fibonacci decomposition
%% of $n$ and study its classification.
%% \begin{itemize}
%% \item Case $low(n)=2$. Hence $n$ isn't 3-odd and we try to prove
%%   $\FG(\FG(n-1)+1)=1+G(G(n-1))$. Theorem \ref{fibpred} implies that
%%   $low(n-1)>3$, and in particular $n-1$ isn't 3-odd.
%%   So the first IH is $\FG(n-1)=G(n-1)$. By theorem
%%   \ref{Gclass1}, we also know that $G(n-1)=G(n)-1$.
%%   So $\FG(n-1)+1=G(n-1)+1=G(n)$
%%   By theorem \ref{Gclass2}, we know that $G(n)$ has an even lowest
%%   rank, so it cannot be 3-odd, and the second IH is
%%   $\FG(\FG(n-1)+1)=G(\FG(n-1)+1)$.
%%   Since $low(G(n))$ is even, we can use theorem \ref{Gclass1} once
%%   more: $G(G(n)-1) = G(G(n))-1$.
%%   Finally: $\FG(\FG(n-1)+1)=G(\FG(n-1)+1)=G(G(n))=1+G(G(n-1))$.
%% \item Case $n$ 3-odd. We're trying to prove
%%   $\FG(\FG(n-1)+1)=G(G(n-1))$. Theorem \ref{fibpred} implies that
%%   $low(n-1) = 2$, and in particular $n-1$ isn't 3-odd.
%%   So the first IH is $\FG(n-1)=G(n-1)$.
%%   By theorem \ref{Gclass1}, we also know that $G(n-1)=G(n)$.
%%   Moreover theorem \ref{Gthree} shows that $\FG(n-1)+1 = G(n)+1$
%%   cannot be 3-odd.
%%   So the second IH gives $\FG(\FG(n-1)+1) = G(\FG(n-1)+1)$.
%%   Now, $low(G(n))=2$ by theorem \ref{Gclass2} so $G(G(n)+1)=G(G(n)$
%%   by theorem \ref{Gclass1}, or equivalently $G(\FG(n-1)+1)=G(G(n-1))$,
%%   hence the desired equation.
%% \item Case $n$ 3-even. As in all the remaining cases, we're now trying here
%%   to prove $\FG(\FG(n-1)+1)=1+G(G(n-1))$.
%%   This case is very similar to the previous one until the point
%%   where we study $G(n)+1$ which is now 3-odd (still thanks to theorem
%%   \ref{Gthree}).
%%   So the second IH gives now
%%   $\FG(\FG(n-1)+1) = 1+G(\FG(n-1)+1)$. And we conclude just as before
%%   by ensuring for the same reasons that $G(\FG(n-1)+1)=G(G(n-1))$.
%% \item Case $low(n)=4$.
%%   Since 4 is even, we also have $G(n-1)=G(n)-1$.
%%   Hence $low(G(n))=3$ by theorem \ref{Glow}, and $G(G(n)-1)=G(G(n))$
%%   and $G(G(n)+1)=1+G(G(n))$, both by theorem \ref{Gclass1}.
%%   Finally $1+G(G(n-1)) = 1+G(G(n)) = G(G(n)+1)$.
%%   Now, to determine whether $n-1$ is 3-odd,
%%   we need to look deeper in the canonical decomposition of
%%   $n = F_4+F_k+\fibrest$.
%%   \begin{itemize}
%%   \item If the second lowest rank $k$ is odd, then
%%     $n-1 = F_3+F_k+\fibrest$ is hence 3-odd, and the first IH is
%%     $\FG(n-1)=1+G(n-1)$. So
%%     $\FG(n-1)+1=G(n)+1$, and this number has an even lowest rank
%%     (theorem \ref{Gclass2}), it cannot be 3-odd, and the second IH
%%     is:
%%     $\FG(\FG(n-1)+1)=G(\FG(n-1)+1)$. And this is known to be equal
%%     to $G(G(n)+1)=1+G(G(n-1))$.
%%   \item Otherwise $k$ is even and $n-1$ is 3-even
%%     and hence not 3-odd, so the first IH is $\FG(n-1)=G(n-1)$.
%%     Moreover, since $n-1$ is 3-even then $\FG(n-1)+1=G(n-1)+1$ is 3-odd by
%%     theorem \ref{Gthree}. So the second IH is:
%%     $\FG(\FG(n-1)+1)=1+G(\FG(n-1)+1)$. And this is known to be equal
%%     to $1+G(G(n-1)+1)=1+G(G(n))=1+G(G(n-1))$.
%%   \end{itemize}
%% \item Case $low(n)>4$ and odd.
%%   By theorem \ref{fibpred}, $low(n-1)=2$ hence the first IH is
%%   $\FG(n-1)=G(n-1)$. We also know that $G(n-1)=G(n)$ by theorem
%%   \ref{Gclass1}, and that $low(\FG(n-1)+1)=low(G(n)+1)$ is even
%%   by theorem \ref{Gclass2}. The second IH is hence:
%%   $\FG(\FG(n-1)+1) = G(\FG(n-1)+1)$. Finally
%%   $G(G(n)+1)=1+G(G(n))$ since $low(G(n))=2\neq 1$.
%% \item Case $low(n)>4$ and even.
%%   In this case, $n-1$ is 3-odd (theorem \ref{threeevenodd}), so
%%   the first IH is $\FG(n-1)=1+G(n-1)$. We also know that
%%   $G(n-1)=G(n)-1$ by theorem \ref{Gclass1}.
%%   By theorem \ref{Glow} we know that $low(G(n))=low(n)-1 > 3$, so
%%   by theorem \ref{fibsucc} $low(G(n)+1)=2$, and
%%   $\FG(n-1)+1=1+G(n)$ cannot be 3-odd : the second IH is hence
%%   $\FG(\FG(n-1)+1)=G(\FG(n-1)+1)$. Moreover $low(G(n))$ is
%%   also known to be odd, so 
%%   by theorem \ref{Gclass1} we have $G(G(n-1))=G(G(n)-1)=G(G(n))$.
%%   The final step is
%%   $G(1+G(n))=1+G(G(n))$ also by theorem \ref{Gclass1} (since $low(G(n))\neq
%%   2$).
%% \end{itemize}
%% \end{proof}

%% As an immediate consequence, $\FG$ is always greater or equal
%% than $G$, but never more than $G+1$. And we've already studied
%% the distance between 3-odd numbers, which is always 5 or 8,
%% while the first 3-odd number is 7. So $\FG$ and $G$ are actually
%% equal more than 80\% of the time.

%% \subsection{$\FG$ and its ``derivative'' $\Delta\FG$}

%% We consider now the ``derivative'' $\Delta\FG$ of $\FG$, defined via
%% $\Delta \FG(n) = \FG(n+1)-\FG(n)$. This study will be quite
%% similar to the corresponding section \ref{deltaG} for $G$.
%% We already know from theorem
%% \ref{FGprops} that the output of $\Delta\FG$ is always either 0 or 1.

%% \begin{theorem}\label{FGdelta}
%%   For all $n>2$, $\Delta\FG(n+1) = 1 - \Delta\FG(n).\Delta\FG(\FG(n+1))$.
%% \end{theorem}
%% \begin{proof}
%% We already know that $\Delta\FG(n)=0$ implies $\Delta\FG(n+1)=1$:
%% we cannot have $\FG(n)=\FG(n+1)=\FG(n+2)$.
%% Consider now some $n>2$ such that
%% $\FG(n+1)-\FG(n)=1$. By using the recursive equation of $\FG$
%% for $n+1>3$ and $n+2>3$, we have as expected
%% \begin{align*}
%%  \Delta\FG(n+1) & = \FG(n+2)-\FG(n+1) \\
%%                 & =(n+3-\FG(\FG(n+1)+1))-(n+2-\FG(\FG(n)+1)) \\
%%                 & = 1 - (\FG(\FG(n+1)+1)-\FG(\FG(n)+1) \\
%%                 & = 1 - (\FG(\FG(n+1)+1)-\FG(\FG(n+1)) \\
%%                 & = 1 - \Delta\FG(\FG(n+1))
%% \end{align*}
%% \end{proof}

%% Note: when compared with theorem \ref{Gdelta} about $\Delta G$, 
%% the equation
%% above looks really similar, but has a different inner call
%% ($\Delta\FG(\FG(n+1))$ instead of $\Delta G(G(n))$), and doesn't
%% hold for $n=2$.

%% For $n>2$, this equation provides a way to express $\FG(n+2)$ in terms of
%% $\FG(n+1)$ and $\FG(n)$ and $\FG(\FG(n+1))$ and $\FG(\FG(n+1)+1)$.
%% For $n>1$, we know that $G(n+1)<n+1$ hence $\FG(n+1)+1<n+2$.
%% It is also clear that $n+1$ and $n$ and $\FG(n+1)$ are all
%% strictly less than $n+2$.
%% So we could use this equation as an alternative way to define
%% recursively $\FG$,
%% alongside  initial equations $\FG(0)=0$ and $\FG(1)=\FG(2)=1$ and
%% $\FG(3)=2$ and $\FG(4)=3$.
%% We proved in Coq that $\FG$ is indeed the unique function to
%% satisfy these equations (see {\tt FD\_unique} and
%% {\tt  fg\_implements\_FD}).

%% \subsection{An alternative recursive equation for $\FG$}

%% During the search for an algebraic definition of $\FG$, we
%% first discovered and proved correct the following result,
%% which unfortunately doesn't uniquely characterize $\FG$.
%% We mention it here nonetheless, for completeness sake.

%% \begin{theorem}
%% For all $n>3$ we have $\FG(n-1) +\FG(\FG(n)) = n$.
%% \end{theorem}
%% \begin{proof}
%% We proceed as for theorem \ref{FGeqn}, except that we
%% use internally the original recursive equation for $G$
%% instead of the alternative equation of theorem \ref{Galt}.
%% We take $n>3$, call $k$ its depth (which is greater or equal to 3),
%% and note $\nn = \flip(n)$. First:
%% $$\FG(\FG(n))=\flip(G(\flip(\flip(G(\nn)))))=\flip(G(G(\nn)))$$
%% The depth of $G(G(\nn))$ is $k-2\neq 0$, so the definition
%% of $\flip$ gives:
%% $$\FG(\FG(n)) = \flip(G(G(\nn))) = 1+F_{k+1}-G(G(\nn))$$
%% Now, thanks to the recursive definition of $G$ at $\nn+1$, we have
%% $$\FG(\FG(n)) = 1+F_{k+1}+G(\nn+1)-\nn-1$$
%% \begin{itemize}
%% \item If $\depth(n-1)=k$ then $\flip(n-1)=\nn+1$ and we
%% conclude by more uses of the $\flip$ definition:
%% $\depth(G(\nn+1))=\depth(G(\flip(n-1))=\depth(n-1)-1=k-1$, so:
%% $$\FG(n-1)=\flip(G(\flip(n-1)))=\flip(G(\nn+1))= 1+F_{k+2}-G(\nn+1)$$
%% And finally:
%% $$\FG(\FG(n)) = F_{k+1} + (1+F_{k+2}-\FG(n-1))-(1+F_{k+3}-n) = n-\FG(n-1)$$

%% \item If $\depth(n-1)\neq k$ then $n$ is the least number at
%% depth $k$, so $n=1+F_{k+1}$, and:
%% $$\FG(n-1) + \FG(\FG(n)) = \FG(F_{k+1})+\FG(\FG(1+F_{k+1})) =
%% F_{k}+(1+F_{k-1}) = n$$
%% \end{itemize}
%% \end{proof}

%% Note that the following function $f:\mathbb{N}\to\mathbb{N}$
%% also satisfies this equation, even with the same initial
%% values than $\FG$:
%% \begin{align*}
%%    f(0) & = 0 \\
%%    f(1) & = f(2)=1  \\
%%    f(3) & = 2 \\
%%    f(4) & =f(5) = 3 \\
%%    f(6) & = 5 \\
%%    f(7) & = 3 \\
%%  \forall n\le 4, ~ f(2n) & = n-2 \\
%%  \forall n\le 4, ~ f(2n+1) & = 4
%% \end{align*}
%% This function isn't monotone. We actually proved in Coq that
%% $\FG$ is the only monotone function that satisfies the previous
%% equation and the initial constraints $0\mapsto 0$, $1\mapsto 1$, $2\mapsto 1$, $3\mapsto
%% 2$ (see {\tt alt\_mono\_unique} and {\tt alt\_mono\_is\_fg}).
%% We will not detail these proofs here, the key ingredient is to
%% prove that any monotone function satisfying these equations
%% will grow by at most 1 at a time.



%% \section{Conclusion}

%% The proofs for theorems \ref{FGeqn} and \ref{comp-fg-g} are
%% surprisingly tricky, all our attempts at simplifying them
%% have been rather unsuccessful. But there's probably still room
%% for improvements here, please let us know if you find or
%% encounter nicer proofs.

%% Perhaps using the definition of $G$
%% via real numbers could help shortening some proofs.
%% We proved in file \doc{Phi} this definition
%% $\forall n, G(n)=\lfloor (n+1)/\varphi\rfloor$ where $\varphi$
%% is the golden ratio $(1+\sqrt{5})/2$. But this has been done
%% quite late in our development, and we haven't tried to use this
%% fact for earlier proofs. Anyway,
%% relating this definition with the flipped function $\FG$
%% doesn't seem obvious.

%% Another approach might be to relate
%% $\FG$ more directly to some kind of Fibonacci decomposition.
%% Of course, now that theorem \ref{comp-fg-g} is proved, we
%% know that $\FG$ shifts the ranks of the Fibonacci decompositions
%% just as $G$, except for 3-odd numbers where a small $+1$
%% correction is needed. But could this fact be established more
%% directly ? For the moment, we are only aware of the following
%% ``easy'' formulation of $\FG$ via decompositions:
%% if $n$ is written as $F_{k+2}-\fibrest$ where $k=\depth(n)$
%% and $\fibrest$ form a canonical decomposition of $F_{k+2}-n$, then
%% $\FG(n)=F_{k+1}-\Sigma F_{i-1}+\epsilon$, where $\epsilon=1$ whenever
%% the decomposition above includes $F_2$, and $\epsilon=0$
%% otherwise. But this statement didn't brought any new insights
%% for the proof of theorem \ref{comp-fg-g}, so we haven't formulated
%% it in Coq.

%% As possible extensions of this work, we might consider later
%% the other recursive functions proposed by Hofstadter :
%% \begin{itemize}
%% \item $H$ defined via $H(n)=n-H(H(H(n-1)))$, or its generalized
%%    version with an arbitrary number of sub-calls instead of 2
%%    for $G$ and 3 for $H$.
%% \item $\overline{H}$, the flipped version of $H$.
%% \item $M$ and $F$, the mutually recursive functions (``Male''
%% and ``Female'').
%% \end{itemize}
%% We've already done on paper a large part of the analysis of
%% $M$ and $F$, and should simply take the time to certify
%% it in Coq. The study of $H$ and $\overline{H}$ remains to be
%% done, it looks like a direct generalization of what we've
%% done here, but surprises are always possible.

\begin{thebibliography}{1}
\bibitem{GEB}
 Hofstadter, Douglas R.,
 {\it Gödel, Escher, Bach: An Eternal Golden Braid},
 1979, Basic Books, Inc, NY.

\bibitem{Coq}
 {\it The Coq Proof Assistant Reference Manual}, 2015,\\
 \mbox{\url{http://coq.inria.fr}}.

\bibitem{OEIS-Fib}
 {\it The On-Line Encyclopedia of Integer Sequences}, Sequence
 A45: Fibonacci numbers, \url{http://oeis.org/A45}

\bibitem{OEIS-Cow}
 {\it The On-Line Encyclopedia of Integer Sequences}, Sequence
 A930: Narayana's cows sequence, \url{http://oeis.org/A930}

\bibitem{OEIS-G}
 {\it The On-Line Encyclopedia of Integer Sequences}, Sequence
 A5206: Hofstadter G-sequence, \url{http://oeis.org/A5206}

\bibitem{OEIS-H}
 {\it The On-Line Encyclopedia of Integer Sequences}, Sequence
 A5374 : Hofstadter H-sequence, \url{http://oeis.org/A5374}

\end{thebibliography}

\end{document}
