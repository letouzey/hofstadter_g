\documentclass[a4paper,11pt]{article}

\usepackage[T1]{fontenc}
\usepackage[utf8x]{inputenc}
\usepackage[english]{babel}
\usepackage{url}
\usepackage{amsthm}

\title{Hofstadter's problem for the curious reader}
\author{Pierre letouzey}

\begin{document}
\newtheorem{theorem}{Theorem}
\newtheorem{definition}{Definition}
\maketitle

\newcommand{\FG}{\ensuremath{\overline{G}}}
\newcommand{\fibrest}{\ensuremath{\Sigma F_i}}

\section{Introduction}
This document summarizes the proofs made during a Coq development in
Summer 2015. This development investigates the function $G$ introduced
by Hofstadter in his famous ``Gödel, Escher, Bach'' book\cite{??}, as
well as the related infinite $G$ tree. The left/right flipped variant
$\FG$ of this $G$ tree has also been studied here, following
Hofstadter ``problem for the curious reader''.
The initial $G$ function is refered as sequence A005206 in
OEIS\cite{??}, while the flipped version $\FG$ is the sequence
A123070.

The detailled and machine-checked proofs can be found in the files
of this development\footnote{See
\url{http://www.pps.univ-paris-diderot.fr/~letouzey/hofstadter_g}}
and can be re-checked by running Coq version 8.4 \cite{??} on it.
In this document, we'll only give some proof sketchs of the
main results of this development. No prior knowledge of Coq is assumed
here.

\section{Prior art and contributions}
Most of the results proved here were already mentionned elsewhere,
e.g. on the OEIS pages for $G$ and $\FG$. These results were certified
in Coq without consulting more the literature, so more elegant proofs
may exist elsewhere. To the best of my knowledge, the main novelties
of this development are:
\begin{enumerate}
\item
  A proof of the recursive
equation for $\FG$ which is currently mentionned as a conjecture in OEIS:

$$\forall n>3, \FG(n) = n+1 - \FG(1+\FG(n-1))$$

\item
  The statement and proof of another equation for $\FG$:

$$\forall n>3, \FG(\FG(n)) + \FG(n-1) = n$$

Although simpler than the previous one, this equation isn't
enough to characterize $\FG$ unless an extra monotonocity
condition on $\FG$ is assumed.

\item
  The statement and proof of a result comparing $\FG$ and $G$:
for all $n$, $\FG(n)$ is either $G(n)+1$ or $G(n)$, depending
on the existence of a decomposition of $n$ as sum of Fibonacci numbers
of the form $F_2 + F_{2p} + ...$ where $F_2 = 2$. Moreover these
specific numbers where $\FG$ and $G$ differ are separated by either
5 or 8.

\item The studies of $G$ and $\FG$ ``derivatives''
  ($\Delta G(n) = G(n+1)-G(n)$ and similarly for $\Delta\FG$),
  leading to yet another characterization of $G$ and $\FG$.

\end{enumerate}

\section{Coq jargon}
For readers without prior knowledge
of Coq but curious enough to have a look at the actual files
of this development, here comes a few words about the Coq syntax.
\begin{itemize}
\item By default, we're manipulating natural numbers only, corresponding
 to Coq type {\tt nat}. Only exception: in file {\tt Phi.v}
 we switch to real numbers to represent the golden ratio.
\item The symbol {\tt S} denotes the successor of natural numbers.
  Hence {\tt S(2)} is the same as {\tt 3}.
\item As in many modern functional languages such as OCaml or Haskell,
  the usage is to skip parenthesis whenever possible, and
  in particular around atomic arguments of functions. Hence
  {\tt S(x)} will rather be written {\tt S x}.
\item The symbol {\tt pred} is the predecessor for natural numbers.
  By convention, {\tt pred 0 = 0}. Similarly, the subtraction on
  natural numbers is rounded: {\tt 3 - 5 = 0}.
\item Coq allows to define custom predicates to express various
  properties of numbers, lists, etc. These custom predicates are
  introduced by the keyword {\tt Inductive}, followed by some
  ``introduction'' rules for the new predicate. In this development
  we use capitalized names for these predicates (e.g. {\tt Delta},
  {\tt Low}, ...).
\item Coq also accepts new definition of recursive functions via
  the command {\tt Fixpoint}. But these definitions should satisfy
  some criterion to guarantee that the new function is well-defined
  and total. So in this development, we also need sometimes to
  define new functions via explicit justification of termination,
  see for instance {\tt norm} in {\tt Fib.v} or {\tt g\_spec} in {\tt FunG.v}.
\end{itemize}

\section{Fibonacci numbers and decompositions}

This section corresponds to file {\tt Fib.v}.

\paragraph{Convention.} In this document and in the Coq development,
we took the following non-standard definition of Fibonacci numbers
$F_n$:

$$F_0 = F_1 = 1$$
$$\forall n,~~ F_{n+2} = F_{n}+F_{n+1}$$

Please note the unusual $F_0 = 1$ instead of $0$. The ranks are hence
shifted by one when compared with, say, OEIS's sequence A000045.
With this choice, we avoid to have to consider zeros in the
decompositions of numbers as sums of Fibonacci numbers (see
below). In Coq, these $F_n$ numbers correspond to the {\tt fib}
function, and we start by proving a few basic properties :
strict positivity, monotony, strict monotony above 1, etc.
We also prove {\tt fib\_inv} which states 
that any positive number can by bounded by
consecutive Fibonacci numbers :
$$\forall n>0, \exists k, F_k \le n < F_{k+1}$$

\subsection{Fibonacci decompositions} The rest of the {\tt Fib.v}
deals with the decompositions of numbers as sums of Fibonacci
numbers.
\begin{definition}
 A decomposition $n = \fibrest$ is said to be \emph{canonical} if:
\begin{itemize}
\item[(a)] $F_0$ isn't in the decomposition
\item[(b)] A Fibonacci number appears at most once in the decomposition
\item[(c)] Fibonacci numbers in the decomposition aren't consecutive
\end{itemize}
A decomposition is said to be \emph{relaxed} if at least conditions
(a) and (b) hold. 
\end{definition}
For instance $11$ admits the canonical decomposition
$F_3+F_5 = 3 + 8$, but also the relaxed decomposition $F_1+F_2+F_3+F_4
= 1 + 2 + 3 + 5$.
On a technical level, we represented in Coq the decompositions
as sorted lists of ranks, and we used a predicate {\tt Delta $p$ $l$}
to express that any element in the list $l$ exceeds the previous
element by at least $p$. We hence use $p=2$ for canonical
decompositions, and $p=1$ for relaxed ones. See file
{\tt DeltaList.v} for the definition and properties of this
predicate.

Then we proved Zeckendorf's theorem (actually discovered earlier by Lekkerkerker):

\begin{theorem}[Zeckendorf]
Any natural number has a unique canonical decomposition.
\end{theorem}

\begin{proof}
The proof of this theorem is quite standard : to decompose
$n$, take the highest $F_k$ which is less or equal to $n$
(cf {\tt fib\_inv} below),
and continue recursively with $n-F_k$. We cannot obtain this way
two consecutive $F_{k+1}$ and $F_k$, otherwise $F_{k+2}$ could have
been used in the first place. And $F_0$ isn't necessary in the
decomposition since we could use $F_1 = 1$ instead.
For proving the unicity, the key ingredient is that this kind of
sum cannot exceed the next Fibonacci number. For instance
$F_1+F_3+F_5 = 1+3+8 = 12 = F_6 - 1$.
\end{proof}

\begin{definition}
For a non-empty decomposition, its \emph{lowest rank} is the rank of
the lowest Fibonacci number in this decomposition. By extension,
the lowest rank $low(n)$ of a number $n\neq 0$ is the lowest
rank of its (unique) canonical decomposition.
\end{definition}
For instance $low(11)=3$. In Coq, $low(n)=k$ is written
via the predicate {\tt Low 2 $n$ $k$}. Note that this notion is
in fact more general : {\tt Low 1 $n$ $k$} says that $n$ can be
decomposed via a relaxed decomposition of lowest rank $k$.

Interestingly, a relaxed decomposition can be transformed into
a canonical one (see Coq function {\tt norm}):

\begin{theorem}[normalisation]
Consider a relaxed decomposition of $n$, made of $k$ terms and
whose lowest rank is $r$. We can build a canonical decomposition
of $n$, made of no more than $k$ terms, and whose lowest rank can
be written $r+2p$ with $p\ge 0$.  
\end{theorem}
\begin{proof}
We simply eliminate the highest consecutive Fibonacci numbers (if any)
by summing them : $F_m+F_{m+1} \to F_{m+2}$, and repeat this
process. By dealing with highest numbers first, we avoid the apparition of
duplicated terms in the decomposition: $F_{m+2}$ wasn't already
in the decomposition, otherwise we would have considered
$F_{m+1}+F_{m+2}$ instead. Hence all the obtained decompositions
during the process are correct relaxed decompositions. The number of
terms is decreasing by 1 at each step, so the process is garanteed to
terminate, and will necessarily stops on a decomposition with no
consecutive Fibonacci numbers : we obtain indeed the canonical
decomposition of $n$. And finally, it is easy to see that the
lowest rank is either left intact or raised by 2 at each step
of the process.
\end{proof}

\subsection{Classifications of Fibonacci decompositions.}
The rest of {\tt Fib.v} deals with classifications of numbers
according to the lowest rank of their canonical decomposition.
In particular, this lowest rank could be 1, 2 or more. It will
also be interesting to distinguish between lowest ranks that
are even or odd. These kind of classifications and their
properties will be eavily used
during theorem\ref{??} which compares $\FG$ and $G$.
For instance:
\begin{theorem}[decomposition of successor]\label{fibsucc}
\noindent
\begin{enumerate}
\item $low(n) = 1$ implies $low(n+1)$ is even.
\item $low(n) = 2$ implies $low(n+1)$ is odd and different from 1.
\item $low(n) > 2$ implies $low(n+1) = 1$.
\end{enumerate}
\end{theorem}
\begin{proof}
Let $r$ be $low(n)$. We can write $n = F_r + \fibrest$ in a canonical
way.
\begin{enumerate}
\item If $r = 1$, then $n+1 = F_2 + \fibrest$ is a relaxed decomposition,
  and we conclude thanks to the previous normalization theorem.
\item If $r = 2$, then $n+1 = F_3 + \fibrest$ is a relaxed decomposition,
  and we conclude similarly.
\item If $r > 2$, then $n+1 = F_1 + F_r + \fibrest$ is directly a
  canonical decomposition.
\end{enumerate}
\end{proof}

We also studied the decomposition of the predecessor
$n-1$ in function of $n$. This decomposition depends of the parity of
the lowest rank in $n$, since for $r\neq 0$ we have:

$$ F_{2r} - 1 = F_1 + F_3 + ... + F_{2r-1}$$
$$ F_{2r+1} - 1 = F_2 + F_4 + ... + F_{2r}$$

Hence:

\begin{theorem}[decomposition of predecessor]
\noindent
\begin{enumerate}
\item If $low(n)$ is even then $low(n-1) = 1$
\item If $low(n)$ is odd and different from 1, then $low(n-1) = 2$
\item For $n>3$, if $low(n)=1$ then $low(n-1)>2$
\end{enumerate}
\end{theorem}
\begin{proof}
\noindent
\begin{enumerate}
\item Let the canonical decomposition of $n$ be $F_{2r} +
  \fibrest$. Then we have $n-1 = F_1 + ... + F_{2r-1} + \fibrest$, which
 is also a canonical decomposition. Moreover $r\neq 0$ since $F_0$
 isn't allowed in decompositions. Hence the decomposition of $n-1$
 contains at least the term $F_1$, and $low(n-1)=1$.
\item When $low(n) = 2r+1$ with $r\neq 0$, we decompose
 similarly $F_{2r+1}-1$, leading to a lowest rank 2 for $n-1$.
\item Finally, when $n = F_1 + \fibrest$, then the canonical decomposition
  of $n-1$ is directly the rest of the decomposition of $n$, which
  cannot starts by $F_1$ nor $F_2$ for canonicity reasons.
\end{enumerate}
\end{proof}

\subsection{Subdivision of decompositions starting by $F_2$}

When $low(n)=2$ and $n>2$, the canonical decomposition of $n$
admits at least a second-lowest term: $n = F_2 + F_k + ...$ and we
will sometimes
need to consider the parity of this second-lowest rank.

\begin{definition}
\noindent
\begin{itemize}
\item A number $n$ is said \emph{2-even} when its canonical decomposition
  is of the form $F_2 + F_{2p} + ...$.
\item A number $n$ is said \emph{2-odd} when its canonical decomposition
  is of the form $F_2 + F_{2p+1} + ...$
\end{itemize}
\end{definition}
In Coq, this corresponds to the {\tt TwoEven} and {\tt TwoOdd}
predicates.
In fact, the 2-even numbers will precisely be the locations where
$G$ and $\FG$ differs (see theorem\ref{??}). The first of these 2-even number is
$7 = F_2+F_4$, followed by $15 = F_2 + F_6$ and $20 = F_2+F_4+F_6$.

A few properties of 2-even and 2-odd numbers:
\begin{theorem}
\noindent
\begin{enumerate}
\item A number $n$ is 2-even if and only if it admits
a \emph{relaxed} decomposition of the form $F_2 + F_{2p}+...$.
\item A number $n$ is 2-even if and only if $low(n)=2$ and
$low(n-2)$ is even.
\item A number $n$ is 2-odd if and only if $low(n)=2$ and
$low(n-2)$ is odd.
\item When $low(n)$ is odd and at least 5, then $n-1$ is 2-even.
\item Two consecutive 2-even numbers are always apart by 5 or 8.
\end{enumerate}
\end{theorem}

\begin{proof}
\noindent
\begin{enumerate}
\item
  The left-to-right implication is obvious, since a canonical
  decomposition can in particular be seen as a relaxed one. Suppose
  now the existence of such a relaxed decomposition. During its
  normalization, the $F_2$ will necessarily be left intact, and the
  term $F_{2p}$ can grow, but only by steps of 2.
\item Once again, the left-to-right implication is
  obvious. Conversely, we start with the canonical decomposition of
  $n-2=F_{2p}+\fibrest$. If $p\neq 1$, this leads to the desired canonical
  decomposition of $n$ as $F_2+F_{2p}+\fibrest$. And the case $p=1$ is
  impossible: assume $p=1$, then we could turn the
  decomposition of $n-2$ into a canonical decomposition of $n-4$
  whose lowest rank is at least 4. This gives us a relaxed
  decomposition of $n$ of the form $F_1+F_3+\fibrest$. After normalization,
  we would obtain that $low(n)$ is odd, which is contradictory with
  $low(n)=2$.
\item We proceed similarly for $low(n)=2$ and $low(n-2)$ odd implies
  that $n$ is 2-odd. First we have a canonical decomposition of
  $n-2 = F_{2p+1} + \fibrest$. Then $p=0$ would give a relaxed
  decomposition of $n$ starting by 3, hence $low(n)$ odd, impossible.
  And $p<>0$ allows to write $n = F_2+F_{2p+1}+\fibrest$ in a canonical way.
\item When $n = F_{2p+1}+\fibrest$ with $p>1$, then $n-1 = F_2+F_4+F_{2p}+\fibrest$
  and the condition on $p$ ensures that the terms $F_2$ and $F_4$ are
  indeed present.
\item If $n$ is 2-even, it could be written $F_2+F_{2p}+\fibrest$.
  When $p>2$, then $n+5 = F_2+F_4+F_{2p}+\fibrest$ is also 2-even.
  When $p=2$, we need to consider the third term in the decomposition
  (if any).
  \begin{itemize}
  \item Either $n$ is of the form $F_2+F_4+F_6+...$
    Then $n+F_5 = F_2 + F_4 + F_7 + ...$ is a relaxed
    decomposition of $n+8$, hence $n+8$ is 2-even (see the first
    part of the current theorem above).
  \item Either $n$ is of the form $F_2+F_4+\fibrest$
    where all terms in $\fibrest$ are strictly greater than $F_6$.
    Then $n+F_5 = F_2 + F_6 + \fibrest$ is a relaxed
    decomposition of $n+8$, hence $n+8$ is 2-even.
  \end{itemize}
  We finish the proof by considering all intermediate numbers between
  $n$ and $n+8$ and we show that $n+5$ is the only one of them that
  might be 2-even:
  \begin{itemize}
  \item $low(n+1)$ is odd (cf. theorem \ref{fibsucc}).
  \item $n+2 = F_1+F_3+F_{2p}+\fibrest$, and normalizing this
    relaxed decomposition shows that $low(n+2)=1$.
  \item $n+3 = F_2+F_3+F_{2p}+\fibrest$, and normalizing this
    relaxed decomposition will either combine $F_3$ with some
    higher terms (leading to an odd second-lowest term and
    hence $n+3$ is 2-odd) or combine $F_3$ with $F_2$ (in which
    case $low(n+3)\ge 4$).
  \item $n+4 = F_1+F_2+F_3+F_{2p}+\fibrest$, hence $low(n+4)$ is
    odd.
  \item $n+6 = F_5+F_{2p}+\fibrest$ is a relaxed decomposition
   whose lowest rank is either 5 (when $p>2$) or 4 (when $p=2$).
   After normalization, we obtain $low(n+6)\ge 4$.
  \item Hence $low(n+7)=1$ by last case of theorem \ref{fibsucc}.
  \end{itemize}
\end{enumerate}
\end{proof}




\section{The $G$ function}

This section corresponds to file {\tt FunG.v}


\section{The $\FG$ function}

This section corresponds to file {\tt FlipG.v}


\end{document}
